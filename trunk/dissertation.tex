%% Select the dissertation mode on
% See the documentation for more information about the available class options
% If you give option 'draft' or 'draft*', the draft mode is set on
\documentclass[dissertation]{aaltoseries}
\usepackage[utf8]{inputenc}
% Lipsum package generates bullshit
\usepackage{lipsum}
% Set the document languages
\usepackage[finnish,swedish,english]{babel}

% The author of the dissertation
\author{Lauri H\"ame}
% The title of the thesis
\title{Demand-Responsive Transport: Models and Algorithms}

\begin{document}

%% The abstract of the dissertation in English
% Use this command!
\draftabstract{
Demand-responsive transport (DRT) is an advanced, user-oriented form of public transport between 
bus and taxi, involving flexible routing of small or medium sized vehicles.
This dissertation presents mathematical models for demand-responsive transport and algorithms
that can be used to solve combinatorial problems related to vehicle routing and journey planning.}
% Let's add another one in Finnish
\draftabstract[finnish]{Demand-responsive transport (DRT) is an advanced, user-oriented form of public transport between 
bus and taxi, involving flexible routing of small or medium sized vehicles.
This dissertation presents mathematical models for demand-responsive transport and algorithms
that can be used to solve combinatorial problems related to vehicle routing and journey planning.}
% And yet another one in Swedish
\draftabstract[swedish]{Demand-responsive transport (DRT) is an advanced, user-oriented form of public transport between 
bus and taxi, involving flexible routing of small or medium sized vehicles.
This dissertation presents mathematical models for demand-responsive transport and algorithms
that can be used to solve combinatorial problems related to vehicle routing and journey planning.}

%% Preface
% If you write this somewhere else than in Helsinki, use the optional location.
%\begin{preface}[Helsinki]
%\lipsum[1-4]
%\end{preface}
\maketitle

%% Table of contents of the dissertation
\tableofcontents

%% For article dissertations, remove if you write a monograph dissertation.
\listofpublications

%% Add lists of figures and tables as you usually.

%% Add list of abbreviations, list of symbols, etc., using your preferred package/method.

%% The main matter, one can obviously use \input or \include

\chapter{Introduction}
\section{Demand-responsive transport today$\ldots$}
Demand-Responsive Transport (DRT) is often referred to as a form of public transport between bus and taxi involving 
flexible routing and scheduling of small or medium sized vehicles. This means that 
the vehicle routes are updated daily or in real time by incorporating information on
the demand for transportation. Usually, the customers of a DRT service are required to
request and book their trips in advance by placing trip requests including information
on the origin and destination of the trip as well as the desired pickup or drop-off time.
The vehicle operator uses this information to provide service in a way that the passenger
needs are satisfied.


DRT systems are typically used to provide transportation in areas with low 
transportation demand, where a regular bus service might not be as efficient. 
Another common application of DRT arises in door-to-door transportation
of elderly or handicapped people (paratransit).
%Paratransit: DRT is available to the general public, whereas paratransit is available to pre-qualified user bases
%share taxis: DRT is pre-booked in advance, whereas share taxis are operated on an ad-hoc basis
%Taxicabs: DRT generally carries more people, and passengers may have less control over their journey on the principle of DRT being a shared[4] system as opposed to an exclusive vehicle for hire. Additionally, journeys may divert en-route for new bookings.[6]
DRT services are often fully or partially funded by local 
authorities, as providers of socially necessary transport. 
%A small fraction 
Most services that are provided by private companies for commercial reasons
are related to transporting passengers between airports and urban areas.
%Demand-responsive transport services are restricted to a certain operating zone.

The implementation of demand-responsive transport is strongly dependent on the target group
or the business concept of the service. In some services, the vehicle routes are built 
freely according to customer requests, whereas 
other services make use of so-called skeleton routes and schedules, that are varied as required. 
%As such, customers are given specific pick-up and drop-off points and time windows for pick-up and drop-off. 
Some DRT systems make use of terminals, at one or both ends of a route, such as an urban center or airport.
In these applications (one-to-many or many-to-one), customers may specify either the origin or destination of the 
desired trip. Some systems provide door-to-door service within a certain service area and others provide 
service between a set of specified stops. 
For example, a DRT service operating in Nurmij\"arvi (Finland) aims to improve the level and
accessibility of services in a sparsely populated area and
to reduce the costs of public transport. The service operates on
a "many-to-many" basis, that is, there are no predefined routes. 
The stop points are located at a maximum of
900 m from origins and destinations. In the case of
special users, the stop points are non-predefined (door-to-door service). 
%What is clear from the foregoing is that there is an extremely wide range of
%applications of the  

%Generally, DRT systems require passengers to request a journey by booking with a central dispatcher, who determines the 
%trip options available given the customer's location and destination.
%The vehicles used in DRT services are generally small minibuses, 
%allowing to provide a near door to door service by being able to use residential streets.

%\subsection{Strengths}
The popularity of demand-responsive transport has recently grown
mainly due to the shortcomings of conventional
bus and taxi services and new technical developments.
In addition, flexible public transport services provided by local authorities and bus operators in
partnerships with employers, stores and leisure centres are thought to help to break down social exclusion \cite{detr}.
%\subsection{Weaknesses}
However, current DRT services have often been
criticised because of their relatively high cost of provision,
their lack of flexibility in route planning and their
inability to manage high demand \cite{mageean}. 

At the present moment, a large number of demand-responsive transport services are 
in operation. Most of such services operate within relatively small neighbourhoods and 
during low-use daytime hours, when there is not enough demand for traditional 
public transport. Thus, while the current services meet their current needs,
demand-responsive transport remains a relatively small business
compared to traditional transportation services, not to speak of private cars.

What if a DRT system was implemented in large scale, in a way that service could be provided
for an entire metropolitan area?

\section{$\ldots$and tomorrow}
Several new ideas and concepts related to demand-responsive transport
services operating in urban areas have been recently presented, see for example \cite{cortes}.
These ideas are often motivated by problems arising from the congestion of urban 
areas caused by the increasing number of private cars.
Thus, one of the main present goals of planning demand-responsive transport is seen
to be the developing of \emph{functional public transport services able to compete
with private car and taxicab traffic}.


The popularity of the private car as a means of transport is partly based on
a direct connection between the origin and destination of a trip and
a short total travel time. In order to compete with private cars, public transport 
should thus offer connections as direct and fast as possible, in which 
the walking and waiting time are minimized. 
Another major advantage of the private car is seen to be the availability
of the car at any time, even without planning beforehand. The study of large scale 
demand-responsive transport has therefore been directed towards highly dynamic services, which allow 
customers to request service not long before they are willing to depart.
In addition, a demand-responsive public transport system should be
able to offer an alternative for transportation without
the inconvenience related to conventional public transport.

%The popularity of the private car as a means of transport is based on
%a direct connection between the origin and destination of a trip and
%a short total travel time. Thus, to compete with private cars, a transportation system 
%should offer connections as direct and fast as possible, in which 
%the walking and waiting times of customers are minimized. 
%In addition, a demand-responsive transport service should be
%able to offer an alternative for transportation without
%the inconvenience related to conventional public transport.

The total travel time in public transport consists of
\emph{walking time} from origin to pick-up point, \emph{waiting time} at
the pick-up point, \emph{ride time}, that is, the time spent in the
vehicle, possible \emph{transfer time} and walking time from 
drop-off point to destination. In order %for this door-to-door travel time
to attract people with private cars, it is necessary that the waiting and 
riding times are within acceptable bounds. In addition, it can be suggested that
the service should be a near door-to-door service and the amount of
transfers between vehicles should be minimal. 

Intuitively, the idea of a large scale DRT system seems promising.
With state-of-the-art engineering, there should be no insuperable technical hindrances
in implementing such a service.

%In the following sections, the strengths and weaknesses of a large scale DRT
%system providing a high level of service are examined.

\subsection{Opportunities}
The fact that demand-responsive transport is "there for you when you want
and where you want" is thought to be a major advantage compared to conventional 
public transport. While it may not be feasible to think that DRT could provide
a level of service substantially better than that offered by taxi cabs, a system that could combine customers'
trips efficiently could be more cost-efficient than a conventional taxi organization.
This would make it possible to provide more inexpensive service without compromising
too much on the level of service experienced by customers.
%In addition, if the trips were aggregated efficiently, and the number of vehicles
%per unit area was large, the waiting times could in fact be slightly shorter compared to current
%taxi services.

Compared to private cars, demand-responsive transport is thought to have several advantages in urban areas.
%For example, the current average occupancy in private cars in the Helsinki metropolitan area is around 1.3.
For a model of a hypothetical large-scale demand-responsive public transport system for the Helsinki 
metropolitan area, simulation results published in 2005 demonstrated that "in an urban area with one 
million inhabitants, trip aggregation could reduce the health, environmental, and other detrimental 
impacts of car traffic typically by 50 - 70\%, and if implemented could attract about half of the car 
passengers, and within a broad operational range would require no public subsidies" \cite{tuomisto}. 
In addition to providing affordable transportation without the additional expenses related 
to maintenance, taxes and parking fees, 
demand-responsive transport could eliminate many other, possibly concealed, concerns related to private cars, 
including the difficulty of finding parking space and the stress related to
driving in hazardous conditions or traffic jams.

At this point, one might ask: If the large scale demand-responsive transport system is superior 
compared to the alternatives, why has it not been implemented in practice? 

\subsection{Possible issues} %Possible issues
%As previously stated, current demand-responsive transport services are often fully or partially funded by local 
%authorities. This is also true for public transport in general. Thus, 
While it is clear is that implementing a large-scale demand-responsive transport system 
would require significant investments, it is not clear 
whether there would be enough demand for such a service were it implemented. 

For example, it might not be realistic nor beneficial from the social point of view 
to think that a conventional heavy rail system was replaced by demand-responsive minibuses, 
due to the high efficiency of heavy rail.
Moreover, traditional public transport in general has many significant advantages compared to
demand-responsive transport: Taking into account the current experience from DRT services,
a major issue can be seen to be the reliability of the service. So far, estimating ride times accurately
in a service with no fixed routes has proven to be somewhat insuperable, not least because
of the human drivers, who are required to follow routes that are constantly changing, and the
differences in their driving styles. Another disadvantage of DRT arises when customers are 
required to book their trips in advance and thus commit themselves to the service or payment
at the time the trip is booked. In traditional bus services, this problem does not
exist since customers may adjust their personal schedules dynamically according to known timetables,
without pre-commitments. A commitment to a trip can be even more binding than in a taxi service:
A normal taxi can wait for some time for the customer, for example, if the customer is at home 
when the taxi arrives, but it might not be reasonable that a demand-responsive minibus with 
customers on board would wait many minutes for one customer with the expense of other customers.

While demand-responsive transport has many rewards compared to the private car as argued above,
the car has many characteristics that are hard to compensate with public services. 
Firstly, a person who has already invested in a car and thus settles yearly taxes and
maintenance costs, is often not willing to use other transportation services since it would
cost more than the marginal cost of using the car.
Secondly, the car is unbeatable in many cases when speaking of flexibility: 
It is available at any time of the day, even without planning beforehand. 
Even if a demand-responsive transport service accepted immediate requests without
a minimum pre-order time, the customer would still be committed to wait for the 
designated vehicle to arrive. Thirdly, the private car is thought to be the most convenient 
way of carrying large amounts of luggage and goods. The car is also often used for
storing equipment, which is not likely to be possible in a public service. 
%Finally, the private car is still thought to give its owner a certain status.   

Despite the above threats related to large scale 
demand-responsive transport, the concept should be studied carefully.
Even if the private car has its advantages in the current state of the world,
it may become practically useless in congested urban areas.
%In addition, environmental issues are given a lot of consideration at the present moment.
%: % in welfare states:  Some are even ready compromise on price and quality if a service proves to be "green". 

%
%
%
%%Conventional public transport, especially heavy rail, is thought to be more
%%beneficial from the social point of view.
%
%For example, demand-responsive minibuses are not likely to be able to compete with heavy rail
%systems
%
%
%The private car,
%as a means of transport, has many advantages  
%
%
%
%
%
%
%
%%Furthermore, in order to achieve efficiency and a good level of service, it 
%%is suggested that the following elements should be given special attention.

\section{Problem statement}
This work is focused on the theoretical problems arising in the planning of
public transport in general and demand-responsive transport in particular. The main goals are i) to develop models for a priori studying 
different forms of transportation services without having to implement them in practice and ii) to
develop algorithms for solving combinatorial problems related to public transport.



\subsection{Vehicle routing}
A vast majority of theoretical studies related to DRT are formalized as combinatorial
optimization problems involving the construction of vehicle routes with respect to
a set of customers, whose pickup and drop-off points are known a priori (see \cite{toth03}).
A large scale system operating in real time, however, induces %imposes 
new challenges: 
%\begin{itemize}

%\item
In order to be able to compete with private cars, service
should be available within a short period of time from the trip request.
This calls for efficient and scalable route construction algorithms, since the modifications 
in vehicle routes have to be executed in real-time.

%\item
In order to ensure a sufficient level of service, the customers' waiting and ride times 
should be relatively limited. The vehicle dispatching algorithms should be designed
in a way that the constrained nature of the problem is taken into account.

%\item
In order to provide efficient stop-to-stop service, %by vehicle mileage and trip duration, 
it is required that the pickup and drop-off points of customers are chosen in a way that 
the vehicle mileage and the excess ride time reflected to customers on board is minimized while 
the walking distance remains within reasonable limits.

\subsection{Journey planning}
A DRT service should be compatible with other transportation
modes. In conventional dial-a-ride services, trip bookings
are made by calling the service provider, but
a modern service should certainly be automated and enable on-line trip booking in order
to attract a large population. This requires a route planner that is capable of 
communicating with conventional public transport as well as DRT.
%\end{itemize}
   

% The remainder of this document is organized as follows. In chapter \ref{potential},
% the potential for passenger pooling and trip combining is evaluated by means of
% a simplified simulation model. In chapter \ref{irdarp}, the problem of dispathcing
% vehicles in a dynamic DRT service is formalized and a solution approach is described.
% Chapter \ref{simutools} introduces a generalized simulation model designed for large scale DRT and
% a collection of simulation results for an unconstrained immediate-request service.
% An adjustable single vehicle routing algorithm, an important subroutine in the proposed 
% solution, designed for constrained problems arising in real-life situations, 
% is described in chapter \ref{eadarp}.


\section{Mathematical highlights}


%% Examples of article references, remove these from your manuscript!
% Uncomment them, if you want to see the results of these commands in this example document

 % Refer to the Journal paper 1 of this example document
%\citepub{j1} \& \cpub{j1} \& \cp{j1} \& \pageref{j1} \& \ref{j1}

% Refer to the Conference paper of this example document
%\citepub[p.~2]{c1} \& \cpub[Sec.~ 1]{c1} \&  \cp[pp.~1--2]{c1} \& \pageref{c1} \& \ref{c1} 



%% The following commands are for article dissertations, remove them if you write a monograph dissertation.

% Errata list, if you have errors in the publications.
%\errata

\bibliographystyle{plain}
\bibliography{vk}


%% The first publication (journal article)
% Set the publication information.
% This command musts to be the first!
\addpublication{Lauri H\"ame}{An adaptive insertion algorithm for the single-vehicle dial-a-ride problem with narrow time windows}{European Journal of Operational Research}{209, p. 11–22}{February}{2011}{Elsevier B.V.}{jeadarp}
% Add the dissertation author's contribution to that publication (the order can be interchanged with \adderrata).
\addcontribution{This article was written by the author.}
% Add the errata of the publication, remove if there are none (the order can be interchanged with \addauthorscontribution).
%\adderrata{This is wrong}
% Add the publication pdf file, the filename is the parameter (must be the last).
\addpublicationpdf{articles/eadarpejor.pdf}

%% The second publication (conference article, note the optional parameter)
% Set the publication information.
\addpublication[conference]{Esa Hyyti\"a, Lauri H\"ame, Aleksi Penttinen, Reijo Sulonen}{Simulation of a Large Scale Dynamic Pickup and Delivery Problem}{SIMUTools}{Malaga, Spain}{March}{2010}{ICST}{csimutools}
% Add the dissertation author's contribution to that publication.
\addcontribution{Parts of this paper were written the author, including most of Section 3 and parts of Section 1. 
The simulations reported in Section 3 were designed and conducted by the author.}
% No errata
% Add the publication pdf file, the filename is the parameter.
\addpublicationpdf{articles/simutools-2010b.pdf}

\addpublication[conference]{Lauri H\"ame, Jani-Pekka Jokinen, Reijo Sulonen}{Modeling a competitive demand-responsive transport market}{Kuhmo Nectar Conference on Transport Economics}{Stockholm, Sweden}{June-July}{2011}{No copyright holder at this moment}{ccompejor}
\addcontribution{The author was the main author of this article, which was nominated for the best student paper award in Kuhmo Nectar 2011.
The article is under review for publication in Economics of Transportation.}
\addpublicationpdf{articles/compejor.pdf}

\addpublication[conference]{Jani-Pekka Jokinen, Lauri H\"ame, Esa Hyyti\"a, Reijo Sulonen}{Simulation Model for a Demand Responsive Transportation Monopoly}{Kuhmo Nectar Conference on Transport Economics}{Stockholm, Sweden}{June-July}{2011}{No copyright holder at this moment}{cmonop_ecotran}
\addcontribution{Parts of this paper were written the author, including major parts of Sections 2 and 3.
The market mechanisms were programmed into the simulation model and the simulations were executed by the author.
The author produced the figures in this paper and the idea of using the simulation model reported in 
"Simulation of a Large Scale Dynamic Pickup and Delivery Problem" to study market mechanims was originally suggested by the author.
This paper is also under review for publication in Economics of Transportation.
}
\addpublicationpdf{articles/monop_ecotran.pdf}

%% The third publication (another journal paper, accepted for publication, note the optional parameter)
% Set the publication information, detailed information can be empty
\addpublication[conference]{Teemu Sihvola, Lauri H\"ame, Reijo Sulonen}{Passenger-Pooling and Trip-Combining Potential of High-Density Demand Responsive
Transport}{Annual Meeting of the Transportation Research Board}{Washington, D.C.}{January}{2010}{Transportation Research Board}{cpooling}
% Add the dissertation author's contribution to that publication.
\addcontribution{The author programmed and executed the simulations reported in this paper.}
% Add the errata of the publication, remove if there are none.
%\adderrata{This is wrong}
% Add the publication pdf file, the filename is the parameter.
\addpublicationpdf{articles/pooling.pdf}


%% The fourth publication (yet another journal paper, submitted for publication, note the optional parameter)
%% Note that you are allowed to use this option only when submitting the dissertation for pre-examination!
% Set the publication information, detailed information is not printed
\addpublication[submitted]{Lauri H\"ame, Harri Hakula}{Dynamic journeying under uncertainty}{Under review for publication in European Journal of Operational Research}{}{20.12.2011}{}{No copyright holder at this moment}{jdjuejor}
\addcontribution{The author was the main author of this article.}
\addpublicationpdf{articles/djuejor2.pdf}

\addpublication[submitted]{Lauri H\"ame, Harri Hakula, Saara Hyv\"onen}{Dynamic journeying in scheduled networks}{Under review for publication in IEEE Transactions on Intelligent Transportation Systems}{}{16.1.2012}{}{No copyright holder at this moment}{jtoits}
\addcontribution{The author was the main author of this article.}
\addpublicationpdf{articles/rbrorl2.pdf}

\addpublication[submitted]{Lauri H\"ame, Harri Hakula}{A Maximum Cluster Algorithm for Checking the
Feasibility of Dial-A-Ride Instances}{Under review for publication in Transportation Science}{}{16.1.2012}{}{No copyright holder at this moment}{jhitsdarpts}
\addcontribution{The author was the main author of this article.}
\addpublicationpdf{articles/hitsdarpts2.pdf}

\addpublication[submitted]{Lauri H\"ame, Harri Hakula}{Routing by Ranking: A Link Analysis Method for the Constrained Dial-A-Ride Problem}
{Under review for publication in Operations Research Letters}{}{16.1.2012}{}{No copyright holder at this moment}{jrbrorl}
\addcontribution{The author was the main author of this article.}
\addpublicationpdf{articles/rbrorl2.pdf}





\end{document}
