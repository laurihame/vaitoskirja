\documentclass{beamer}
% Class options include: notes, handout, trans
%                        
\usepackage[finnish]{babel}
% Theme for beamer presentation.
\usepackage{beamerthemesplit,xmpmulti,cancel, multirow} 

% Other themes include: beamerthemebars, beamerthemelined,beamerthemetree, beamerthemeplain
\usepackage[utf8]{inputenc}
\usefonttheme{professionalfonts}

\title[Aalto-yliopiston perustieteiden korkeakoulu]{Kysyntäohjautuvan joukkoliikenteen matemaattisia malleja ja algoritmeja}
%\subtitle{Examples of lists, columns and graphics}    % Enter your title between curly braces
\author[L. Häme]{Lauri Häme}                 % Enter your name between curly braces
\institute[Aalto-yliopiston perustieteiden korkeakoulu]{Aalto-yliopiston perustieteiden korkeakoulu}      % Enter your institute name between curly braces
\date{\today}      % Enter the date or \today between curly braces

\usepackage{graphicx}

    \setbeamerfont{section title}{parent=title}
    \setbeamercolor{section title}{parent=titlelike}
    \defbeamertemplate*{section pages}{default}[1][]
    {
      \centering
        \begin{beamercolorbox}[sep=8pt,center,#1]{section title}
          \usebeamerfont{section title}\insertsection\par
        \end{beamercolorbox}
    }
    \newcommand*{\secpage}{\usebeamertemplate*{section pages}}
    
\begin{document}

% Creates title page of slide show using above information
\begin{frame}
  \titlepage
\end{frame}
%\cancele{Talk for 30 minutes} % Add notes to yourself that will be displayed when typeset with the notes class option.

%\section{Johdanto}
% Creates table of contents slide incorporating all \section and \subsection commands.
%\begin{frame}
%  \tableofcontents
%\end{frame}

%\subsection{Johdanto}
\begin{frame}
  \frametitle{Johdanto}   % Insert frame title between curly braces
  \begin{itemize}
    \item 
Kysyntäohjautuva joukkoliikenne = bussi- ja taksipalvelujen välimuoto, joka perustuu ajoneuvojen joustavaan reititykseen
\begin{itemize}
 \item 
 Matkat tilataan esim. internetistä ja ajoneuvojen reitit muodostuvat reaaliaikaisesti tilausten perusteella
 \end{itemize}
 
  \item 
Väitöskirjassa tarkastellaan kolmea teemaa
 \begin{itemize}
\item
Ajoneuvojen reitinlaskenta
\item
Matkustajien matkansuunnittelu
\item
Taloudellinen tasapaino
\end{itemize}
%  \column{2.5in}
 \end{itemize}

 
 
 
 
 
 
 
 
 
%\framebox{\includegraphics[scale=0.6]{esitys01}}
 
%  \end{columns}
\end{frame}

\section{Reitinlaskenta}
\frame{\secpage}
\subsection{Ongelman määrittely}
\begin{frame}
  \frametitle{Kauppamatkustajan ongelma}   % Insert frame title between curly braces
  \begin{itemize}
    \item 
Tunnetuin reitinlaskentaongelma on ns. kauppamatkustajan ongelma (Traveling Salesman Problem, TSP)
\begin{itemize}
\item
Joukko maantieteellisiä pisteitä, joiden väliset etäisyydet tunnetaan
\item
Tavoitteena on löytää lyhin reitti joka kulkee kaikkien pisteiden kautta
\item
Laskennallisesti haastava ongelma
    \end{itemize}
    \end{itemize}
    \end{frame}

    
    \begin{frame}
  \frametitle{Kauppamatkustajan ongelma, esimerkki}   % Insert frame title between curly braces
\centering
\includegraphics[scale=0.25]{tspdemo01}
    \end{frame}
    
        \begin{frame}
  \frametitle{Kauppamatkustajan ongelma, esimerkki}   % Insert frame title between curly braces
\centering
\includegraphics[scale=0.25]{tspdemo02}
    \end{frame}
    

%\subsection{Kysyntäohjautuva joukkoliikenne}
% \begin{frame}
%     \frametitle{Reitinlaskenta kuljetuspalveluissa}
%     \begin{itemize}
%     \item
%     Käytännössä, esim. kuljetuspalveluissa, reitinlaskentaongelma on usein monimutkaisempi
%     \item
%     Rajoituksia
%     \begin{itemize}
% \item 
% Kapasiteetti - Ajoneuvoihin mahtuu vain tietty määrä tavaraa/ matkustajia kerrallaan
% \item
% Aika - Kuljetus ei saa kestää liian kauan
% \item
% Edeltävyys - Esim. noutopisteessä pitää käydä ennen toimituspistettä
% \end{itemize}
% \item
% Tavoitefunktio: Lyhin reitti ei välttämättä ole paras
% \item
% Ajoneuvoja eli laskettavia reittejä voi olla useita
% \begin{itemize}
%  \item 
%  Ajoneuvojen ja matkustajien yhdistely
% \end{itemize}
%   \end{itemize}
% 
% \end{frame}


\begin{frame}
    \frametitle{Reitinlaskenta kuljetuspalveluissa}
    \begin{itemize}
    \item
    Kuljetuspalveluissa reitinlaskentaongelma on monimutkaisempi
    \item
    Ongelma voi olla staattinen tai dynaaminen
    \item
    Tavoitteita: tilausten lukumäärä, kustannukset, palvelutaso
%     \begin{itemize}
%      \item 
%      tilausten lukumäärän maksimointi
%      \item
%      kustannusten minimointi
%      \item
%      asiakkaiden palvelutason optimointi
%     \end{itemize}
    \item
    Rajoituksia: aika, kapasiteetti, edeltävyys
%     \begin{itemize}
% \item 
% Kapasiteetti - Ajoneuvoihin mahtuu vain tietty määrä tavaraa/ matkustajia kerrallaan
% \item
% Aika - Kuljetus ei saa kestää liian kauan
% \item
% Edeltävyys - Esim. noutopisteessä pitää käydä ennen toimituspistettä
% \end{itemize}
% \item
% Ajoneuvoja eli laskettavia reittejä voi olla useita
% \begin{itemize}
%  \item 
%  Ajoneuvojen ja matkustajien yhdistely
% \end{itemize}
  \end{itemize}
  \begin{center}
%       \includegraphics[scale=0.5]<1>{maxcesim04}
%       \includegraphics[scale=0.5]<2>{maxcesim03}
%       \includegraphics[scale=0.5]<3>{maxcesim02}
      \includegraphics[scale=0.4]<1>{vrp}
      \includegraphics[scale=0.4]<2>{vrp2}
      \end{center}
\end{frame}








\begin{frame}
  \frametitle{Reitinlaskenta kysyntäohjautuvassa joukkoliikenteessä}   % Insert frame title between curly braces
  \begin{columns}[c]
  \column{3.5in}  % slides are 3in high by 5in wide
  \begin{itemize}
    %\item 
    %Kysyntäohjautuva joukkoliikenne perustuu pienten tai keskisuurten ajoneuvojen (esim. minibussien) joustavaan reititykseen 
    \item
    Asiakkaat voivat tilata matkoja reaaliaikaisesti esim. internet-käyttöliittymällä
    \item
    Ajoneuvojen reitit muodostuvat tilattujen matkojen perusteella
    \item
    Kunkin matkatilauksen yhteydessä ratkaistaan reitinlaskentaongelma
    %kaksi tehtävää:
    %\begin{itemize}
    %\item
    %Ajoneuvon valinta 
    %\item
    %Valitun ajoneuvon reitin optimointi
    %\end{itemize}
  \end{itemize}
    \column{1.5in}
\centering

%\framebox{
\includegraphics[scale=0.8]{tilauskaavio02}
%}
 
  \end{columns}
\end{frame}

        \begin{frame}
  \frametitle{Kysyntäohjautuva joukkoliikenne, 1 ajoneuvo, esimerkki }   % Insert frame title between curly braces
\begin{center}
\includegraphics[scale=0.3]{ekademo01}
\end{center}
\end{frame}

        \begin{frame}
  \frametitle{Kysyntäohjautuva joukkoliikenne, 1 ajoneuvo, esimerkki}   % Insert frame title between curly braces
\begin{center}
\includegraphics[scale=0.3]{ekademo02}
\end{center}
    \end{frame}
    
            \begin{frame}
  \frametitle{Kysyntäohjautuva joukkoliikenne, 1 ajoneuvo, esimerkki}   % Insert frame title between curly braces
\begin{center}
\includegraphics[scale=0.3]{ekademo03}
\end{center}
    \end{frame}
    
            \begin{frame}
  \frametitle{Kysyntäohjautuva joukkoliikenne, 1 ajoneuvo, esimerkki}   % Insert frame title between curly braces
\begin{center}
\includegraphics[scale=0.3]{ekademo04}
\end{center}
    \end{frame}
    
            \begin{frame}
  \frametitle{Kysyntäohjautuva joukkoliikenne, 1 ajoneuvo, esimerkki}   % Insert frame title between curly braces
\begin{center}
\includegraphics[scale=0.3]{ekademo05}
\end{center}
    \end{frame}
    
            \begin{frame}
  \frametitle{Kysyntäohjautuva joukkoliikenne, 1 ajoneuvo, esimerkki}   % Insert frame title between curly braces
\begin{center}
\includegraphics[scale=0.3]{ekademo06}
\end{center}
    \end{frame}

    
\begin{frame}
\frametitle{Aikarajoitteet}
\begin{itemize}
 \item 
 Matka-aika voi pidentyä yllättäen reittimuutosten johdosta
 \item
 Palvelutasosta voidaan huolehtia aikarajoitteilla
 \begin{itemize}
  \item 
  Esim. lähtö aikaisintaan klo 12:00, perillä viimeistään klo 13:00.
  %\item 
  %Aikarajoitteet voivat olla osittain asiakkaan ja osittain järjestelmän määrittämiä
 \end{itemize}
 \item
  Käytetään reitinlaskennassa, jotta tietty minimipalvelutaso toteutuisi
 \item
 Liian tiukat rajoitteet vähentävät reitin joustavuutta
\end{itemize}
\begin{center}
\includegraphics[scale=0.8]{aikaikkuna01}
\end{center}
\end{frame}    
    
    
    
\begin{frame}
  \frametitle{Ajoneuvon ja reitin valintaongelma}   % Insert frame title between curly braces
\begin{itemize}
 \item 
 Usean ajoneuvon tapauksessa jokaiselle uudelle asiakkaalle valitaan ajoneuvo ja valitulle ajoneuvolle määrätään uusi reitti
  \item
 Ajoneuvon ja reitin valinnassa pitää ottaa huomioon mm.
 \begin{itemize}
  \item 
  Uuden asiakkaan aiheuttama reitin pitenemä
  \item
  Uuden asiakkaan palvelutaso ja muille asiakkaille aiheutuva palvelutason muutos
  \item
  Kysyntäennuste
 \end{itemize}
%\end{itemize}
%\begin{center}
% \includegraphics[scale=0.6]{vskuvatavoite}
%\end{center}
\item
Yleisesti jos minimoidaan reitin pituutta, palvelutaso saattaa kärsiä ja jos optimoidaan ainoastaan palvelutasoa, kustannukset kasvavat
%\begin{itemize}
\end{itemize}
\end{frame}

\subsection{Hajautettu ratkaisu}
\begin{frame}
  \frametitle{Hajautettu ratkaisu}   % Insert frame title between curly braces
\begin{itemize}
\item
Yritetään lisätä uusi asiakas johonkin olemassaolevista reiteistä %$\to$ käsitellään jokainen ajoneuvo erikseen
%\item
%Valitaan se ajoneuvo, jonka reitille uusi asiakas sopii parhaiten
\item
Lasketaan jokaiselle ajoneuvolle uusi reittiehdotus ja valitaan niistä paras/parhaat
\item
Ajoneuvojen reittiehdotukset lasketaan erikseen, toisistaan riippumatta 
\begin{itemize}
 \item 
 Rinnakkaislaskenta
\end{itemize}
%\item
%Asiakkaalle voidaan ilmoittaa ajoneuvon tunniste tilauksen yhteydessä
\end{itemize}
\end{frame}



                \begin{frame}
  \frametitle{Hajautettu ratkaisu, esimerkki}   % Insert frame title between curly braces
\begin{minipage}[t][0.3\textheight][t]{\textwidth}
  \begin{itemize}
 \item 
 Kaksi ajoneuvoa, joista toinen odottaa tyhjänä
\end{itemize}
  \end{minipage}
  \vfill
  \begin{minipage}{\textwidth}
    \centering
\includegraphics[scale=0.6]{valinta01}
  \end{minipage}
    \end{frame}
    
    
                    \begin{frame}
  \frametitle{Hajautettu ratkaisu, esimerkki}   % Insert frame title between curly braces
\begin{minipage}[t][0.3\textheight][t]{\textwidth}
  \begin{itemize}
 \item 
 Kaksi ajoneuvoa, joista toinen odottaa tyhjänä
   \item 
 Uusi asiakas tilaa matkan ($u^+,u^-$)
\end{itemize}
  \end{minipage}
  \vfill
  \begin{minipage}{\textwidth}
    \centering
\includegraphics[scale=0.6]{valinta02}
  \end{minipage}
  
  \end{frame}
    
                        \begin{frame}
  \frametitle{Hajautettu ratkaisu, esimerkki}   % Insert frame title between curly braces
\begin{minipage}[t][0.3\textheight][t]{\textwidth}
  \begin{itemize}
 \item 
 Kaksi ajoneuvoa, joista toinen odottaa tyhjänä
   \item 
 Uusi asiakas tilaa matkan ($u^+,u^-$)
 \item
 Ehdotus 1: reitin pitenemä minimoituu, palvelutaso kärsii
\end{itemize}
  \end{minipage}
  \vfill
  \begin{minipage}{\textwidth}
    \centering
\includegraphics[scale=0.6]{valinta03}
  \end{minipage}
  
  \end{frame}
  
  
                          \begin{frame}
  \frametitle{Hajautettu ratkaisu, esimerkki}   % Insert frame title between curly braces
\begin{minipage}[t][0.3\textheight][t]{\textwidth}
  \begin{itemize}
 \item 
 Kaksi ajoneuvoa, joista toinen odottaa tyhjänä
   \item 
 Uusi asiakas tilaa matkan ($u^+,u^-$)
 \item
 Ehdotus 1: reitin pitenemä minimoituu, palvelutaso kärsii
  \item
 Ehdotus 2: palvelutaso on paras mahdollinen, reitin pituus kasvaa enemmän
\end{itemize}
  \end{minipage}
  \vfill
  \begin{minipage}{\textwidth}
    \centering
\includegraphics[scale=0.6]{valinta04}
  \end{minipage}
  
  \end{frame}
    

    \begin{frame}
\frametitle{Yksinkertainen lisäysalgoritmi (Insertion algorithm)}
 \begin{itemize}
\item
Yksinkertainen ratkaisu reittiehdotusten laskemiselle on lisätä uuden asiakkaan nouto- ja toimituspiste sopivaan väliin 
\item
Ei-täydellinen ratkaisu: olemassaolevien pisteiden järjestys säilyy
\end{itemize}
\begin{center}
 \includegraphics[scale=0.5]<1>{insertion02}
  \includegraphics[scale=0.5]<2>{insertion03}
   \includegraphics[scale=0.5]<3>{insertion04}
    \includegraphics[scale=0.5]<4>{insertion05}
      \includegraphics[scale=0.5]<5>{insertion06}
       \includegraphics[scale=0.5]<6>{insertion07}
        \includegraphics[scale=0.5]<7>{insertion08}
         \includegraphics[scale=0.5]<8>{insertion09}

           
\end{center}

\end{frame}
    
    \begin{frame}
\frametitle{Laajennettu lisäysalgoritmi (Adaptive insertion algorithm)}
 \begin{itemize}
\item
Rakennetaan lisäysperiaatteella rinnakkain useampi vaihoehtoinen reitti ja valitaan niistä paras
\end{itemize}
\begin{center}
 \includegraphics[scale=0.5]<1>{insertion02}
  \includegraphics[scale=0.5]<2>{insertion03}
   \includegraphics[scale=0.5]<3>{insertion04}
    \includegraphics[scale=0.5]<4>{insertion05}
     \includegraphics[scale=0.5]<5>{insertion06}
      \includegraphics[scale=0.5]<6>{insertion07}
       \includegraphics[scale=0.5]<7>{insertion08}
       \includegraphics[scale=0.5]<8>{insertion09}
\\
\hfill
\\
\hfill
\\
                \includegraphics[scale=0.5]<1>{insertion02}   
                \includegraphics[scale=0.5]<2>{insertion03}
                \includegraphics[scale=0.5]<3>{insertion04}  
                \includegraphics[scale=0.5]<4>{insertion05}
                \includegraphics[scale=0.5]<5>{insertion06b} 
                \includegraphics[scale=0.5]<6>{insertion07b} 
                \includegraphics[scale=0.5]<7>{insertion08b} 
                \includegraphics[scale=0.5]<8>{insertion09b} 
\end{center}

\end{frame}


    \begin{frame}
\frametitle{Täydellinen lisäysalgoritmi (Exact insertion algorithm)}
 \begin{itemize}
 \item
Rakennetaan lisäysperiaatteella rinnakkain kaikki mahdolliset reitit (enintään $\frac{(2n)!}{2^n}$ kpl)
\item
Osa reiteistä voidaan hylätä rajoitusten perusteella
%\item
%Perusidea: luetellaan yhden ajoneuvon kaikki mahdolliset reitit ja valitaan niistä paras 
%\item
%Kaikki mahdolliset reitit ($\frac{(2n)!}{2^n}$ kpl) saadaan lisäämällä asiakkaat yksi kerrallaan \emph{kaikkiin edellisiin} reitteihin
\end{itemize}
\hfill \\
  \begin{columns}[c]
  \column{0.7in}
  \column{2.5in}  % slides are 3in high by 5in wide
 $1^+,1^-$ \\
 \hfill \\
  $1^+,1^-,2^+,2^-$ \\
    $1^+,2^+,1^-,2^-$ \\
     $1^+,2^+,2^-,1^-$ \\
      $2^+,1^+,1^-,2^-$ \\
      $2^+,1^+,2^-,1^-$ \\
      $2^+,2^-,1^+,1^-$ \\
      \column{2.5in}
      {\tiny 
        $1^+,1^-,2^+,2^-,3^+,3^-$ \\
        $1^+,1^-,2^+,3^+,2^-,3^-$ \\
        $1^+,1^-,3^+,2^+,2^-,3^-$ \\
        $1^+,3^+,1^-,2^+,2^-,3^-$ \\
        $3^+,1^+,1^-,2^+,2^-,3^-$ \\
        
        $1^+,1^-,2^+,3^+,3^-,2^-$ \\
        $1^+,1^-,3^+,2^+,3^-,2^-$ \\
        $1^+,3^+,1^-,2^+,3^-,2^-$ \\
        $3^+,1^+,1^-,2^+,3^-,2^-$ \\
        
        $1^+,1^-,3^+,3^-,2^+,2^-$ \\
        $1^+,3^+,1^-,3^-,2^+,2^-$ \\
        $3^+,1^+,1^-,3^-,2^+,2^-$ \\
        
      \ldots
      }
      \end{columns}
\end{frame}
    
    
    
    \begin{frame}
\frametitle{Täydellinen lisäysalgoritmi (Exact insertion algorithm)}
 \begin{itemize}
 \item
Rakennetaan lisäysperiaatteella rinnakkain kaikki mahdolliset reitit (enintään $\frac{(2n)!}{2^n}$ kpl)
\item
Osa reiteistä voidaan hylätä rajoitusten perusteella
%\item
%Perusidea: luetellaan yhden ajoneuvon kaikki mahdolliset reitit ja valitaan niistä paras 
%\item
%Kaikki mahdolliset reitit ($\frac{(2n)!}{2^n}$ kpl) saadaan lisäämällä asiakkaat yksi kerrallaan \emph{kaikkiin edellisiin} reitteihin
\end{itemize}
\hfill \\
  \begin{columns}[c]
  \column{0.7in}
  \column{2.5in}  % slides are 3in high by 5in wide
 $1^+,1^-$ \\
 \hfill \\
  $1^+,1^-,2^+,2^-$ \\
    $\cancel{1^+,2^+,1^-,2^-}$ \\
     $1^+,2^+,2^-,1^-$ \\
      $\cancel{2^+,1^+,1^-,2^-}$ \\
      $\cancel{2^+,1^+,2^-,1^-}$ \\
      $2^+,2^-,1^+,1^-$ \\
      \column{2.5in}
      {\tiny 
        $1^+,1^-,2^+,2^-,3^+,3^-$ \\
        $1^+,1^-,2^+,3^+,2^-,3^-$ \\
        $\cancel{1^+,1^-,3^+,2^+,2^-,3^-}$ \\
        $\cancel{1^+,3^+,1^-,2^+,2^-,3^-}$ \\
        $3^+,1^+,1^-,2^+,2^-,3^-$ \\
        
        $1^+,1^-,2^+,3^+,3^-,2^-$ \\
        $1^+,1^-,3^+,2^+,3^-,2^-$ \\
        $\cancel{1^+,3^+,1^-,2^+,3^-,2^-}$ \\
        $3^+,1^+,1^-,2^+,3^-,2^-$ \\
        
        $\cancel{1^+,1^-,3^+,3^-,2^+,2^-}$ \\
        $1^+,3^+,1^-,3^-,2^+,2^-$ \\
        $3^+,1^+,1^-,3^-,2^+,2^-$ \\
        \
      \ldots
      }
      \end{columns}
\end{frame}    
    
    
    
    
    
    
    
    
    
        \begin{frame}
\frametitle{Lisäysalgoritmi, tuloksia}
 \begin{itemize}
\item
Tiukkojen aika- tai kapasiteettirajoitusten vallitessa kaikkien mahdollisten reittien lukumäärä pysyy kohtuullisena ja 
täydellinen algoritmi tuottaa nopeasti optimaalisen ratkaisun
\item
Jos rajoitukset eivät ole tiukkoja, saadaan tehokas ratkaisu
rajoittamalla rinnakkaisten reittien lukumäärää
\end{itemize}
\hfill \\
\begin{center}
\includegraphics[scale=0.7]{vskuva}
\end{center}
\begin{itemize}
 \item 
 Yksinkertainen lisäysalgoritmi on täydellinen, kun $n < 3$
\end{itemize}

\end{frame}
    
    
    
    
        \begin{frame}
\frametitle{Lisäysalgoritmi, jatkotutkimus}
 \begin{itemize}
\item
Millä ehdoilla yksinkertainen lisäysalgoritmi tuottaa optimaalisen ratkaisun? Kuinka suuri on virhe?
\item
Mikä heuristinen kustannusfunktio tuottaa käyvän ratkaisun suurimmalla todennäköisyydellä, kun säilytetään vain osa reiteistä?
\begin{itemize}
 \item 
 Numeeriset tulokset: Total time slack, max-min time slack, route duration
\end{itemize}
\end{itemize}

\end{frame}
    
    
    
    
    
\subsection{Keskitetty ratkaisu}
\begin{frame}
  \frametitle{Keskitetty ratkaisu}   % Insert frame title between curly braces
\begin{itemize}
\item
Uuden matkatilauksen saapuessa etsitään parasta mahdollista asiakkaiden, ajoneuvojen ja reittien yhdistelmää
\item
Toistaiseksi noutamattomien asiakkaiden ajoneuvo voi vaihtua
\item
Periaate sisältää hajautetut ratkaisut
\end{itemize}
\end{frame}

\begin{frame}
  \frametitle{Keskitetty ratkaisu, esimerkki 1} 
  \begin{minipage}{\textwidth}
    \centering
    \fbox{\includegraphics[scale=0.5]{valinta02}} \\
    \hfill \\
    \hfill \\
\fbox{\includegraphics[scale=0.5]{valinta05}}
  \end{minipage}
\end{frame}    
    
\begin{frame}
  \frametitle{Keskitetty ratkaisu, esimerkki 2} 
  \begin{itemize}
   \item 
   Keskitetyn ratkaisun merkitys korostuu rajoitetuissa tapauksissa 
  \end{itemize}
    \begin{center}
    \includegraphics[scale=0.7]<1>{keskitettyesim01}
    \includegraphics[scale=0.7]<2>{keskitettyesim02} 
    \includegraphics[scale=0.7]<3>{keskitettyesim03} 
    \includegraphics[scale=0.7]<4>{keskitettyesim04} 
    \end{center}
\end{frame}    

\begin{frame}
  \frametitle{Maksimiklusteriperiaate (Maximum cluster algorithm)} 
  \begin{itemize}
   \item 
   Perusidea: Etsitään suurin asiakasjoukko (klusteri), joka sopii yhden ajoneuvon reitille 
   \item
   Uuden matkatilauksen saapuessa klusterit lasketaan uudelleen
  \end{itemize}
  \begin{center}
      \includegraphics[scale=0.5]<1>{maxcesim04}
      \includegraphics[scale=0.5]<2>{maxcesim03}
      \includegraphics[scale=0.5]<3>{maxcesim02}
      \includegraphics[scale=0.5]<4>{maxcesim01}
      \end{center}
\end{frame}  


\begin{frame}
  \frametitle{Arvojärjestysmenetelmä (Routing by Ranking)} 
  \begin{itemize}
   \item 
   Maksimiklusterit voidaan määrittää tehokkaasti järjestämällä nouto- ja toimituspisteet arvojärjestykseen
   \item
   Suurimman arvon saavat pisteet, joista on mahdollista siirtyä mahdollisimman moneen arvokkaaseen pisteeseen aikarajojen sisällä, $h_i = \sum_{j \in N_i} h_j$
   \item
   Arvojärjestys saadaan laskemalla suurinta ominaisarvoa vastaava ominaisvektori (ks. HITS-hakualgoritmi)
  \end{itemize}
  \begin{center}
      \includegraphics[scale=0.5]<1>{hub01}
      \includegraphics[scale=0.5]<2>{hub02}
      \includegraphics[scale=0.5]<3>{hub03}
      \end{center}
\end{frame}  

\begin{frame}
  \frametitle{Arvojärjestysmenetelmä (Routing by Ranking)} 
  \begin{itemize}
   \item 
Sink graph: = DAG, jossa yhdelle solmulle (päätepiste) on lisätty silmukka
\item
Dominantin ominaisvektorin arvot vastaavat polkujen lukumäärää eri pisteistä päätepisteisiin
\end{itemize}
\begin{columns}
 \column{2.5in}
\begin{center}
{\scriptsize
\begin{align*}
& \left(
\begin{array}{ccccccc}
 0 & 1 & 0 & 0 & 1 & 1 & 1 \\
 0 & 0 & 0 & 0 & 1 & 1 & 1 \\
 0 & 0 & 0 & 1 & 1 & 1 & 1 \\
 0 & 0 & 0 & 0 & 0 & 1 & 1 \\
 0 & 0 & 0 & 0 & 0 & 1 & 1 \\
 0 & 0 & 0 & 0 & 0 & 0 & 1 \\
 0 & 0 & 0 & 0 & 0 & 0 & 1 \\
\end{array}
\right)\\
\\
& h = (8,4,6,2,2,1,1)
\end{align*} 
}
\end{center}

\column{2.5in}
\begin{center}
      \includegraphics[scale=0.5]{hub01}
      \end{center}
      \end{columns}

\end{frame}  



\begin{frame}
  \frametitle{Keskitetty ratkaisu, tuloksia} 
  \begin{itemize}
   \item 
    Arvojärjestysmenetelmä tuottaa tehokkaasti käypiä ratkaisuja tiukkojen rajoitusten vallitessa 
    \item
    Kertaluokkaa nopeampi aikaisempiin menetelmiin verrattuna
    \item
    Yleisesti keskitetyn ratkaisun merkitys korostuu, kun
    \begin{itemize}
     \item 
     rajoitteet ovat tiukkoja
     \item
     reitit ovat pitkiä (pitkät ennakkotilausajat)
    \end{itemize}
    \item
    Testiaineistossa ratkaisematon ongelma: 5 ajoneuvoa, 84 asiakasta, onko olemassa käypä ratkaisu?
    \item
    Jatkotutkimus: Arvojärjestysmenetelmän soveltaminen muihin ongelmiin
   \end{itemize}
\end{frame}  





% \begin{frame}
%   \frametitle{Hajautetun ja keskitetyn ratkaisun vertailu} 
%   \begin{itemize}
%    \item 
% Hajautettu ratkaisu 
% \begin{itemize}
% \item
% Laskennallisesti kevyempi
% \item
% Palvelutason kannalta luotettavampi
% \end{itemize}
% 
% \item
% Keskitetty ratkaisu
%    \begin{itemize}
% \item
% Parempi kustannustehokkuus
% \item
% Ajoneuvon vaihto ennen noutoa lisää palvelutason epävarmuutta
% \end{itemize}
%    \end{itemize}
% \end{frame}  





\section{Matkansuunnittelu}
\frame{\secpage}
\subsection{Matkansuunnittelun mallit}
\begin{frame}
  \frametitle{Matkansuunnittelu} 
  \begin{itemize}
   \item 
    Matkansuunnittelu (Journey planning) = joukkoliikennevälineen ja reitin valinta
    \item
    Tarkoituksena on löytää matkustajalle paras reitti ja aikataulu lähtöpisteestä määränpäähän, esim.
    \begin{itemize}
     \item 
     16:27: kävely pysäkille A,
     \item
     16:39: bussi numero 58 pysäkiltä A pysäkille B
     \item
     16:53: kävely pysäkiltä B määränpäähän, perillä klo 17:11
    \end{itemize}
   \end{itemize}
     \begin{center}
      \includegraphics[scale=0.2]{reittiopas01}
      \end{center}
\end{frame} 

\begin{frame}
  \frametitle{Deterministinen ja stokastinen malli} 
  \begin{itemize}
   \item 
    \emph{Deterministisillä} menetelmillä voidaan laskea etukäteen paras reitti esim. matka-ajan, odotusajan, kävelymatkan tai vaihtojen lukumäärän suhteen 
    \item
    Todellisuudessa etukäteen laskettu reitti ei välttämättä toteudu esim. myöhästymisien tai vuorojen peruutuksien takia
    \item
    \emph{Stokastinen} malli ottaa huomioon mahdolliset reittimuutokset matkan varrella
    \item
    Mallin avulla voidaan laskea parhaan reitin lisäksi paras matkastrategia eri tavoitteiden suhteen
   \end{itemize}
     \begin{center}
      \end{center}
\end{frame} 

\begin{frame}
  \frametitle{Deterministinen malli, esimerkki} 
  \begin{minipage}[t][0.3\textheight][t]{\textwidth}
  \begin{itemize}
   \item 
Tavoitteena on saapua mahdollisimman aikaisin määränpäähän, vaihto pysäkillä A tai B
\item
Kaikki bussilinjat kulkevat 20 minuutin välein
\item
Nopein matka: bussi 1 klo 15:00, vaihto pysäkillä A, perillä klo 15:30
   \end{itemize}
   \end{minipage}
   \vfill
     \begin{minipage}{\textwidth}
     \begin{center}
     \includegraphics[scale=0.6]{matkansuunnittelu01}
      \end{center}
      \end{minipage}
\end{frame} 

\begin{frame}
  \frametitle{Stokastinen malli, esimerkki} 
  \begin{minipage}[t][0.3\textheight][t]{\textwidth}
  \begin{itemize}
   \item 
Lisätään malliin vaihtojen onnistumisien todennäköisyydet 
\item
Jos kuljetaan busseilla 1 ja 3 pysäkin A kautta, saapumisajan odotusarvo on 15:40
\item
Paras matkastrategia: bussilla 2 pysäkille B, vaihto seuraavaksi saapuvaan bussiin, odotettu saapumisaika 15:37
   \end{itemize}
   \end{minipage}
   \vfill
     \begin{minipage}{\textwidth}
     \begin{center}
     \includegraphics[scale=0.6]{matkansuunnittelu02}
      \end{center}
      \end{minipage}
\end{frame} 


\subsection{Stokastinen malli}
\begin{frame}
  \frametitle{Matka-ajat stokastisessa mallissa} 
  \begin{itemize}
   \item 
    Liikennepalvelujen arvioidut ohitusajat pysäkeillä määritellään satunnaismuuttujina (esim. gammajakauma) odotusarvojen sijaan
\end{itemize}
     \begin{center}
     \includegraphics[scale=0.7]{stokvsdet02}
      \end{center}
    \end{frame} 
    
    
    \begin{frame}
  \frametitle{Vaihdon onnistumisen todennäköisyys} 
  \begin{itemize}
   \item 
    Matka voidaan esittää etappeina, joista jokaisella on alkamis- ja päättymisaika
    \begin{itemize}
     \item 
     $\tau_i:=$ etapin $i$ alkamisaika
     \item 
     $\tau_i':=$ etapin $i$ päättymisaika
    \end{itemize}
    \item
    Vaihto etapilta $i$ etapille $j$ onnistuu todennäköisyydellä $p_{ij} = P(\tau_i' \leq \tau_j)$
\end{itemize}
     \begin{center}
     \includegraphics[scale=0.7]{transferprob02}
      \end{center}
    \end{frame} 

    
            \begin{frame}
  \frametitle{Todennäköisyyksien ehdollisuus matkansuunnittelussa} 
  \begin{itemize}
       \item
    Onnistuneet vaihdot eivät ole riippumattomia tapahtumia
        \begin{itemize}
     \item 
     Esim. Jos bussit 1 ja 2 lähtevät aikataulun mukaan samaan aikaan, on todennäköistä että 
     vaihto onnistuu joko kumpaan tahansa tai ei kumpaankaan
    \end{itemize}
   \item 
    Toteutuneet vaihdot vaikuttavat tulevien vaihtojen onnistumiseen
    \begin{itemize}
     \item 
     Esim. Jos tiukka vaihto bussista 3 bussiin 4 on onnistunut, on todennäköistä että bussi 4 on ollut myöhässä
    \end{itemize}
\end{itemize}
     \begin{center}
     \includegraphics[scale=0.65]{stateexample}
      \end{center}
    \end{frame}  
    
    
    
        \begin{frame}
  \frametitle{Markov-päätösprosessi (Markov Decision Process, MDP)} 
  \begin{itemize}
   \item 
    Matka voidaan esittää Markov-päätösprosessina etappien verkossa
    \item
    \emph{Tilat} ovat etappeja ja \emph{toiminnat} matkustajan valintoja 
    \item
    Tietty toiminta tietyssä tilassa johtaa toiseen tilaan siirtymiseen
    \item
    Jokaiselle tilalle voidaan määrittää optimaalinen valinta (=matkastrategia, optimaalinen politiikka) 
\end{itemize}
     \begin{center}
     \includegraphics[scale=0.55]{walking01c}
      \end{center}
    \end{frame} 
    
    
    \begin{frame}
  \frametitle{Markov-päätösprosessi (Markov Decision Process, MDP)} 
  \begin{itemize}
   \item 
Toiminta määritellään seuraavien etappien ``preferenssijärjestyksenä''
\item
Poikkeustapauksissa voidaan laskea politiikka uudelleen, jolloin preferenssijärjestys voi muuttua
\end{itemize}

\begin{center}
{\scriptsize
\begin{tabular}{p{4cm}c|p{2.5cm}|p{2.5cm}|}
\cline{3-4}
\multirow{5}{*}{} 
\multirow{5}{*}{
\includegraphics[width=0.3\columnwidth]{rankingexample.pdf} 
} 
& & Ranking of successors of leg $42$ & End of leg $42$: transfer options are revealed \\
\cline{3-4}
& & \ \ \ \ \ \ \ \ $1: 32$ & \ \ \ \ \ \ \ \ $\cancel{1: \ 32 }$  \\
& & \ \ \ \ \ \ \ \ $2: 65$ & \ \ \ \ \ \ \ \ $\cancel{2: \ 65 }$  \\
& & \ \ \ \ \ \ \ \ $3: 12$ & \ \ \ \ \ \ \ \ $3: 12$ $\leftarrow$ \\
& & \ \ \ \ \ \ \ \ $4: 54$ & \ \ \ \ \ \ \ \ $\cancel{4: \ 54 }$  \\
& & \ \ \ \ \ \ \ \ $5: 24$ & \ \ \ \ \ \ \ \ $5: 24$  \\
\cline{3-4}
\end{tabular}
}
\end{center}

\end{frame} 
        
    
    
    
    \subsection{Tuloksia}
        \begin{frame}
  \frametitle{Stokastinen matkansuunnittelu, tuloksia} 
  \begin{itemize}
   \item 
Stokastisen matkansuunnittelun merkitys korostuu kun 
\begin{itemize}
 \item 
vaihtojen lukumäärä on suuri 
\item
vaihtoihin liittyy epävarmuutta
\item
halutaan maksimoida matkan luotettavuutta
\end{itemize}
\item
Laskentaa voidaan tehostaa yksinkertaistamalla todennäköisyysmallia
\begin{itemize}
 \item 
Vaihtojen onnistumisien riippumattomuus
\item
Riippumattomuus matkahistoriasta eli aiempien vaihtojen onnistumisesta
\end{itemize}
\item
Millä ehdoilla optimaalinen politiikka säilyy (approksimaatioiden tarkkuus)?
\end{itemize}

    \end{frame}     
    
    
\section{Taloudellinen tasapaino}
\frame{\secpage}
\subsection{Mallin kuvaus}
\begin{frame}
  \frametitle{Taloudellinen tasapaino} 
  \begin{itemize}
   \item 
   Taloudellisessa tarkastelussa otetaan samanaikaisesti huomioon sekä matkustajien että liikennöitsijän päätökset
   \item
   Tasapaino = kysynnän ja tarjonnan kohtaamispiste liikenneverkossa
   \end{itemize}
    \end{frame}   
    

% \begin{frame}
%   \frametitle{Liikennepalvelujen verkkomallit}   % Insert frame title between curly braces
%   \begin{columns}[c]
%   \column{2.5in}  % slides are 3in high by 5in wide
%   \begin{itemize}
% \item
% Perinteinen joukkoliikenne noudattaa kiinteitä reittejä ja aikatauluja
% \item
% Taksit tuottavat suoria matkoja tilausten perusteella
% \item
% Kysyntäohjautuvassa joukkoliikenteessä kokonaisia reittejä muodostetaan reaaliaikaisesti kysynnän mukaan
% \end{itemize}
%   \column{2.5in}
% \framebox{\multiinclude[<+>][format=pdf,graphics={scale=0.5}]{kuvat/bustaxiexample}}
%   \end{columns}
% \end{frame}
    
    
 
    \begin{frame}
  \frametitle{Kysyntäohjautuva joukkoliikenne stokastisena prosessina}   % Insert frame title between curly braces
\begin{itemize}
 \item 
 Kysyntäohjautuvaa joukkoliikennettä voidaan kuvata stokastisena prosessina, joka muodostuu
 \emph{tiloista} ja tilojen välisistä \emph{siirtymätodennäköisyyksistä}
 \item
 Ajoneuvon tila $\approx$ reitti %, kyydissä olevat matkustajat ja sovitut matkat
\end{itemize}
\begin{center}
 \includegraphics[scale=0.65]{tilat01}
\end{center}
\end{frame}


\begin{frame}
  \frametitle{Tilan tarkempi määritelmä}   % Insert frame title between curly braces
  \begin{itemize}
\item
Tilan määrittelee
\begin{itemize}
 \item 
 Reitti ja ajoneuvon sijainti reitillä
 \item
 Kyydissä olevien matkustajien nouto- ja toimituspisteet
 \item
 Sovittujen matkojen nouto- ja toimituspisteet
\end{itemize}
\end{itemize}
\begin{center}
 \includegraphics<1>[scale=0.65]{tilamaar01}
  \includegraphics<2>[scale=0.65]{tilamaar02}
   \includegraphics<3>[scale=0.65]{tilamaar03}
    \includegraphics<4>[scale=0.65]{tilamaar04}
     \includegraphics<5>[scale=0.65]{tilamaar05}
\end{center}

\end{frame}    

\begin{frame}
  \frametitle{Tilan tarkempi määritelmä}   % Insert frame title between curly braces
  \begin{itemize}
\item
Tilan määrittelee
\begin{itemize}
 \item 
 Reitti ja ajoneuvon sijainti reitillä
 \item
 Kyydissä olevien matkustajien nouto- ja toimituspisteet
 \item
 Sovittujen matkojen nouto- ja toimituspisteet
\end{itemize}
\end{itemize}
\begin{center}
 \includegraphics[scale=0.55]{statekuva01}
\end{center}

\end{frame}    
    



    

\subsection{Reititysstrategia ja kysyntä}
\begin{frame}
  \frametitle{Reititysstrategia}   % Insert frame title between curly braces
\begin{itemize}
 \item 
 Tilojen $r$ ja $s$ välinen siirtymätodennäköisyys $p_{rs}$ kuvaa kuinka suuri osuus ajoneuvoista
 siirtyy reitille $s$ kuljettuaan reitin $r$
 \item
Siirtymätodennäköisyydet määrittelevät ns. reititysstrategian, joka johtaa tiettyyn ajoneuvojen tasapainojakaumaan 
\end{itemize}
\begin{center}
 \includegraphics<1>[scale=0.65]{tilat02}
  \includegraphics<2>[scale=0.65]{tilat03}
\end{center}

\end{frame}


\begin{frame}
  \frametitle{Palvelutaso ja kysyntä}   % Insert frame title between curly braces
\begin{itemize}
 \item 
Tietty reititysstrategia tuottaa tietyn palvelutason eri matkoille
 \item
Matkojen kysyntä määräytyy palvelutason ja vaihtoehtoisten kulkumuotojen mukaan
\item
Logit-valintamallissa matkustusvaihtoehdon $i$ utiliteetti muodostuu tunnetusta osasta $V_i$ ja satunnaisesta osasta $\epsilon_i \sim \rm{Gumbel}$
\item
Todennäköisyys valinnalle $i$:
\begin{align*}
 P_i = \frac{e^{V_i}}{\sum_j e^{V_j}}
\end{align*}
%\item
%Kysyntä vaikuttaa optimaaliseen reititysstrategiaan
\end{itemize}
\end{frame}

 
\begin{frame}
  \frametitle{Taloudellinen optimi}   % Insert frame title between curly braces
\begin{itemize}
\item
Mallin avulla voidaan optimoida ajoneuvojen lukumäärä ja matkojen hinnoittelu
\end{itemize}
\begin{center}
 \includegraphics[scale=0.3]{a-optima02}
\end{center}
\end{frame}



\subsection{Tuloksia}
\begin{frame}
  \frametitle{Taloudellinen tasapaino, tuloksia}   % Insert frame title between curly braces
\begin{itemize}
\item
Analyyttinen malli, jolla voidaan kuvata kysyntäohjautuvaa joukkoliikennettä 
\item
Mikrosimulointiin verrattuna etuna on skaalautuvuus ajoneuvojen ja matkustajien lukumäärän suhteen
\item
Mallin sovelluksia:
\begin{itemize}
 \item 
Optimaalisen reititysstrategian, hinnoittelun ja ajoneuvojen lukumäärän määrittäminen eri tilanteissa
\item
Säännöstelyn vaikutusten tutkiminen 
\end{itemize}
\end{itemize}
\end{frame}


\begin{frame}
  \frametitle{Taloudellinen tasapaino, jatkotutkimus}   % Insert frame title between curly braces
\begin{itemize}
\item
Peliteoria ja yleinen tasapainoteoria 
\begin{itemize}
 \item 
 Yleisessä tasapainoteoriassa ``pelaajien'' lukumäärä on niin suuri, että yksi asiakas ei voi vaikuttaa hintaan
\end{itemize}
\item
Mallin soveltaminen Kutsuplus-datalla
\end{itemize}
\end{frame}



\section{Tulosten tarkastelu}
\frame{\secpage}
\begin{frame}
  \frametitle{Tulosten tarkastelu}   % Insert frame title between curly braces
\begin{itemize}
\item
Reitinlaskennassa tulee ottaa huomioon sekä kustannukset että palvelutaso: paras kokonaisratkaisu löytyy kahden optimin välistä
\item
Suuri kysyntä mahdollistaa hyvän palvelutason tuottamisen tehokkaasti
\item
Pidemmät ennakkotilausajat mahdollistavat tarkemman optimoinnin
%Kysyntäohjautuvan joukkoliikenteen pitäisi olla osa olemassaolevaa joukkoliikennejärjestelmää 
%\begin{itemize}
\end{itemize}
\end{frame}
\begin{frame}
  \frametitle{Tulosten tarkastelu}   % Insert frame title between curly braces
\begin{itemize}
 \item 
 Matkansuunnittelun avulla voidaan liittää kysyntäohjautuva palvelu olemassaolevaan joukkoliikennejärjestelmään
 \begin{itemize}
  \item 
 Vaihdolliset yhteydet (sisäiset ja kulkumuotojen väliset vaihdot) 
 %\item
 %``Liftaus'' - valmiiksi laskettujen reittien suora hyödyntäminen 
 \end{itemize}
 \end{itemize}
  \begin{center}
  \includegraphics<1>[scale=0.25]{vaihto01}
    \includegraphics<2>[scale=0.25]{vaihto02}
      \end{center}
\end{frame}


\begin{frame}
  \frametitle{Tulosten tarkastelu}   % Insert frame title between curly braces
\begin{itemize}
 \item 
Joukkoliikenteen lisäksi reitinlaskenta- ja matkansuunnittelumenetelmiä voidaan hyödyntää 
\begin{itemize}
\item
rahti- ja lentoliikenteessä
\item
lähetti- ja ruoankuljetuspalveluissa 
\item
sotilaslogistiikassa
 \end{itemize}
 \item
 Menetelmät soveltuvat erityisesti
 \begin{itemize}
  \item 
  tehtäviin, joihin liittyy rajoitusehtoja (aika, kapasiteetti)
  \item
  luotettavuuden optimointiin
 \end{itemize}
 \end{itemize}
\end{frame}



\begin{frame}
  \frametitle{Tulosten tarkastelu}   % Insert frame title between curly braces
\begin{itemize}
 \item 
Helsingin seudun liikenne käynnisti vuoden 2013 alussa kaikille avoimen kysyntäohjautuvan joukkoliikennepalvelun
\item
Enintään 60 minuutin ennakkotilausaika
\item
Useita erihintaisia matkavaihtoehtoja
\item
Toimii noin 10 kilometrin säteellä Helsingin keskustasta
\item
10 minibussia, määrää kasvatetaan
\item
Teoria on käytäntöä edellä
\begin{itemize}
 \item 
 Reitinlaskennan merkitys korostuu suurissa järjestelmissä
\end{itemize}
 \end{itemize}
\end{frame}


\begin{frame}
  \frametitle{Jatkotutkimus: TSP:n kulmat}   % Insert frame title between curly braces
\begin{itemize}
\item
$N$ pistettä tasossa, $S =$ pisteiden kautta kulkeva lyhin polku 
 \item 
 $S$ avoin polku: onko kulmien keskiarvo aina $\geq 60^{\circ}$? 
  \item 
 $S$ suljettu polku: onko kulmien keskiarvo aina $\geq \frac{n}{n-2} 60^{\circ} $? 
 \end{itemize}
   \begin{center}
  \includegraphics<1>[scale=0.4]{tspdn01}
      \end{center}
\end{frame}

\begin{frame}
  \frametitle{Jatkotutkimus: Kaupunkipyöräongelma}   % Insert frame title between curly braces
\begin{itemize}
\item
Vuokrattavien polkupyörien jakauman tasoitus kuljetusajoneuvoilla
\item
Polkupyörät kerääntyvät tiettyihin paikkoihin tiettyinä aikoina
\item
Data: asemilla olevien polkupyörien lukumäärä 15 min välein, puolen vuoden ajalta (Barcelona)
 \end{itemize}
\end{frame}



% \begin{frame}
%   \frametitle{Muita tutkimusmenetelmiä}   % Insert frame title between curly braces
% \begin{itemize}
%  \item 
%  Kyselytutkimukset
%  \begin{itemize}
%  \item
% Hinnoittelu, maksuhalukkuus
%  \end{itemize}
% \item
% Mikrosimulointi
% \begin{itemize}
%  \item 
%  Algoritmien testaus
%  \item
%  Skaalaedut
% \end{itemize}
% \item
% Jonomallit
%  \end{itemize}
% \end{frame}
    
\end{document}