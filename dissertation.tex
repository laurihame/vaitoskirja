%% Select the dissertation mode on
% See the documentation for more information about the available class options
% If you give option 'draft' or 'draft*', the draft mode is set on
\documentclass[dissertation,draft*]{aaltoseries}
\usepackage[utf8]{inputenc}
% Lipsum package generates bullshit
\usepackage{amsthm, amssymb, amsmath,natbib, algorithm, algorithmic, amsxtra, txfonts}
% Set the document languages
\usepackage[finnish,swedish,english]{babel}

\newtheorem{theorem}{Theorem}
\newtheorem{lemma}[theorem]{Lemma}
\newtheorem*{definition}{Definition}


% The author of the dissertation
\author{Lauri H\"ame}
% The title of the thesis
\title{Demand-Responsive Transport: Models and Algorithms}

\begin{document}

%% The abstract of the dissertation in English
% Use this command!
\draftabstract{
Demand-responsive transport (DRT) is an advanced, user-oriented form of public transport between 
bus and taxi, involving flexible routing of small or medium sized vehicles.
This dissertation presents mathematical models for demand-responsive transport and algorithms
that can be used to solve combinatorial problems related to vehicle routing and journey planning.}
% Let's add another one in Finnish
\draftabstract[finnish]{Demand-responsive transport (DRT) is an advanced, user-oriented form of public transport between 
bus and taxi, involving flexible routing of small or medium sized vehicles.
This dissertation presents mathematical models for demand-responsive transport and algorithms
that can be used to solve combinatorial problems related to vehicle routing and journey planning.}
% And yet another one in Swedish
\draftabstract[swedish]{Demand-responsive transport (DRT) is an advanced, user-oriented form of public transport between 
bus and taxi, involving flexible routing of small or medium sized vehicles.
This dissertation presents mathematical models for demand-responsive transport and algorithms
that can be used to solve combinatorial problems related to vehicle routing and journey planning.}

%% Preface
% If you write this somewhere else than in Helsinki, use the optional location.
%\begin{preface}[Helsinki]
%\lipsum[1-4]
%\end{preface}
\maketitle

%% Table of contents of the dissertation
\tableofcontents

%% For article dissertations, remove if you write a monograph dissertation.
\listofpublications

%% Add lists of figures and tables as you usually.

%% Add list of abbreviations, list of symbols, etc., using your preferred package/method.

%% The main matter, one can obviously use \input or \include

\chapter{Introduction}
\section{Demand-responsive transport today$\ldots$}
Demand-Responsive Transport (DRT) is often referred to as a form of public transport between bus and taxi involving 
flexible routing and scheduling of small or medium sized vehicles. This means that 
the vehicle routes are updated daily or in real time by incorporating information on
the demand for transportation. Usually, the customers of a DRT service are required to
request and book their trips in advance by placing trip requests including information
on the origin and destination of the trip as well as the desired pick-up or drop-off time.
The vehicle operator uses this information to provide service in a way that the passenger
needs are satisfied.


DRT systems are typically used to provide transportation in areas with low 
transportation demand, where a regular bus service might not be as efficient. 
Another common application of DRT arises in door-to-door transportation
of elderly or handicapped people (paratransit).
%Paratransit: DRT is available to the general public, whereas paratransit is available to pre-qualified user bases
%share taxis: DRT is pre-booked in advance, whereas share taxis are operated on an ad-hoc basis
%Taxicabs: DRT generally carries more people, and passengers may have less control over their journey on the principle of DRT being a shared[4] system as opposed to an exclusive vehicle for hire. Additionally, journeys may divert en-route for new bookings.[6]
DRT services are often fully or partially funded by local 
authorities, as providers of socially necessary transport. 
%A small fraction 
Most services that are provided by private companies for commercial reasons
are related to transporting passengers between airports and urban areas.
%Demand-responsive transport services are restricted to a certain operating zone.

The implementation of demand-responsive transport is strongly dependent on the target group
or the business concept of the service. In some services, the vehicle routes are built 
freely according to customer requests, whereas 
other services make use of so-called skeleton routes and schedules, that are varied as required. 
%As such, customers are given specific pick-up and drop-off points and time windows for pick-up and drop-off. 
Some DRT systems make use of terminals, at one or both ends of a route, such as an urban center or airport.
In these applications (one-to-many or many-to-one), customers may specify either the origin or destination of the 
desired trip. Some systems provide door-to-door service within a certain service area and others provide 
service between a set of specified stops. 
For example, a DRT service operating in Nurmij\"arvi (Finland) aims to improve the level and
accessibility of services in a sparsely populated area and
to reduce the costs of public transport. The service operates on
a "many-to-many" basis, that is, there are no predefined routes. 
The stop points are located at a maximum of
900 m from origins and destinations. In the case of
special users, the stop points are non-predefined (door-to-door service). 
%What is clear from the foregoing is that there is an extremely wide range of
%applications of the  

%Generally, DRT systems require passengers to request a journey by booking with a central dispatcher, who determines the 
%trip options available given the customer's location and destination.
%The vehicles used in DRT services are generally small minibuses, 
%allowing to provide a near door to door service by being able to use residential streets.

%\subsection{Strengths}
The popularity of demand-responsive transport has recently grown
mainly due to the shortcomings of conventional
bus and taxi services and new technical developments.
In addition, flexible public transport services provided by local authorities and bus operators in
partnerships with employers, stores and leisure centres are thought to help to break down social exclusion \cite{detr}.
%\subsection{Weaknesses}
However, current DRT services have often been
criticised because of their relatively high cost of provision,
their lack of flexibility in route planning and their
inability to manage high demand \cite{mageean}. 

At the present moment, a large number of demand-responsive transport services are 
in operation. Most of such services operate within relatively small neighbourhoods and 
during low-use daytime hours, when there is not enough demand for traditional 
public transport. Thus, while the current services meet their current needs,
demand-responsive transport remains a relatively small business
compared to traditional transportation services, not to speak of private cars.

What if a DRT system was implemented in large scale, in a way that service could be provided
for an entire metropolitan area?

\section{$\ldots$and tomorrow}
Several new ideas and concepts related to demand-responsive transport
services operating in urban areas have been recently presented, see for example \cite{cortes}.
These ideas are often motivated by problems arising from the congestion of urban 
areas caused by the increasing number of private cars.
Thus, one of the main present goals of planning demand-responsive transport is seen
to be the developing of \emph{functional public transport services able to compete
with private car and taxicab traffic}.


The popularity of the private car as a means of transport is partly based on
a direct connection between the origin and destination of a trip and
a short total travel time. In order to compete with private cars, public transport 
should thus offer connections as direct and fast as possible, in which 
the walking and waiting time are minimized. 
Another major advantage of the private car is seen to be the availability
of the car at any time, even without planning beforehand. The study of large scale 
demand-responsive transport has therefore been directed towards highly dynamic services, which allow 
customers to request service not long before they are willing to depart.
In addition, a demand-responsive public transport system should be
able to offer an alternative for transportation without
the inconvenience related to conventional public transport.

%The popularity of the private car as a means of transport is based on
%a direct connection between the origin and destination of a trip and
%a short total travel time. Thus, to compete with private cars, a transportation system 
%should offer connections as direct and fast as possible, in which 
%the walking and waiting times of customers are minimized. 
%In addition, a demand-responsive transport service should be
%able to offer an alternative for transportation without
%the inconvenience related to conventional public transport.

The total travel time in public transport consists of
\emph{walking time} from origin to pick-up point, \emph{waiting time} at
the pick-up point, \emph{ride time}, that is, the time spent in the
vehicle, possible \emph{transfer time} and walking time from 
drop-off point to destination. In order %for this door-to-door travel time
to attract people with private cars, it is necessary that the waiting and 
riding times are within acceptable bounds. In addition, it can be suggested that
the service should be a near door-to-door service and the amount of
transfers between vehicles should be minimal. 

Intuitively, the idea of a large scale DRT system seems promising.
With state-of-the-art engineering, there should be no insuperable technical hindrances
in implementing such a service.

%In the following sections, the strengths and weaknesses of a large scale DRT
%system providing a high level of service are examined.

\subsection{Opportunities}
The fact that demand-responsive transport is "there for you when you want
and where you want" is thought to be a major advantage compared to conventional 
public transport. While it may not be feasible to think that DRT could provide
a level of service substantially better than that offered by taxi cabs, a system that could combine customers'
trips efficiently could be more cost-efficient than a conventional taxi organization.
This would make it possible to provide more inexpensive service without compromising
too much on the level of service experienced by customers.
%In addition, if the trips were aggregated efficiently, and the number of vehicles
%per unit area was large, the waiting times could in fact be slightly shorter compared to current
%taxi services.

Compared to private cars, demand-responsive transport is thought to have several advantages in urban areas.
%For example, the current average occupancy in private cars in the Helsinki metropolitan area is around 1.3.
For a model of a hypothetical large-scale demand-responsive public transport system for the Helsinki 
metropolitan area, simulation results published in 2005 demonstrated that "in an urban area with one 
million inhabitants, trip aggregation could reduce the health, environmental, and other detrimental 
impacts of car traffic typically by 50 - 70\%, and if implemented could attract about half of the car 
passengers, and within a broad operational range would require no public subsidies" \cite{tuomisto}. 
In addition to providing affordable transportation without the additional expenses related 
to maintenance, taxes and parking fees, 
demand-responsive transport could eliminate many other, possibly concealed, concerns related to private cars, 
including the difficulty of finding parking space and the stress related to
driving in hazardous conditions or traffic jams.

At this point, one might ask: If the large scale demand-responsive transport system is superior 
compared to the alternatives, why has it not been implemented in practice? 

\subsection{Possible issues} %Possible issues
%As previously stated, current demand-responsive transport services are often fully or partially funded by local 
%authorities. This is also true for public transport in general. Thus, 
While it is clear is that implementing a large-scale demand-responsive transport system 
would require significant investments, it is not clear 
whether there would be enough demand for such a service were it implemented. 

For example, it might not be realistic nor beneficial from the social point of view 
to think that a conventional heavy rail system was replaced by demand-responsive minibuses, 
due to the high efficiency of heavy rail.
Moreover, traditional public transport in general has many significant advantages compared to
demand-responsive transport: Taking into account the current experience from DRT services,
a major issue can be seen to be the reliability of the service. So far, estimating ride times accurately
in a service with no fixed routes has proven to be somewhat insuperable, not least because
of the human drivers, who are required to follow routes that are constantly changing, and the
differences in their driving styles. Another disadvantage of DRT arises when customers are 
required to book their trips in advance and thus commit themselves to the service or payment
at the time the trip is booked. In traditional bus services, this problem does not
exist since customers may adjust their personal schedules dynamically according to known timetables,
without pre-commitments. A commitment to a trip can be even more binding than in a taxi service:
A normal taxi can wait for some time for the customer, for example, if the customer is at home 
when the taxi arrives, but it might not be reasonable that a demand-responsive minibus with 
customers on board would wait many minutes for one customer with the expense of other customers.

While demand-responsive transport has many rewards compared to the private car as argued above,
the car has many characteristics that are hard to compensate with public services. 
Firstly, a person who has already invested in a car and thus settles yearly taxes and
maintenance costs, is often not willing to use other transportation services since it would
cost more than the marginal cost of using the car.
Secondly, the car is unbeatable in many cases when speaking of flexibility: 
It is available at any time of the day, even without planning beforehand. 
Even if a demand-responsive transport service accepted immediate requests without
a minimum pre-order time, the customer would still be committed to wait for the 
designated vehicle to arrive. Thirdly, the private car is thought to be the most convenient 
way of carrying large amounts of luggage and goods. The car is also often used for
storing equipment, which is not likely to be possible in a public service. 
%Finally, the private car is still thought to give its owner a certain status.   

Despite the above threats related to large scale 
demand-responsive transport, the concept should be studied carefully.
Even if the private car has its advantages in the current state of the world,
it may become practically useless in congested urban areas.
%In addition, environmental issues are given a lot of consideration at the present moment.
%: % in welfare states:  Some are even ready compromise on price and quality if a service proves to be "green". 

%
%
%
%%Conventional public transport, especially heavy rail, is thought to be more
%%beneficial from the social point of view.
%
%For example, demand-responsive minibuses are not likely to be able to compete with heavy rail
%systems
%
%
%The private car,
%as a means of transport, has many advantages  
%
%
%
%
%
%
%
%%Furthermore, in order to achieve efficiency and a good level of service, it 
%%is suggested that the following elements should be given special attention.

\section{Problem statement}
This work is focused on the discrete and combinatorial problems arising in the planning of
public transport in general and demand-responsive transport in particular. The main goals are (i) to develop models for a priori studying 
different forms of transportation services without having to implement them in practice and (ii) to
develop algorithms for solving combinatorial problems related to public transport.

The following chapters present three approaches to the demand and supply of public transport: 
%First, we assume that the demand is fixed and optimize the supply. Second, for a given supply,
%More precisely, the main problems are stated as follows:

In Chapter \ref{vehiclerouting}: \emph{Vehicle routing}, the demand for transportation is assumed to be known at the individual level. 
Using a fleet of vehicles, what is the best way to satisfy the demand? 

In Chapter \ref{journeyplanning}: \emph{Journey planning}, the schedules and routes of transport services are assumed to be known during a specific time horizon.
Using the scheduled transport services, what is the best way for a commuter to travel from a given origin to a given destination?

In Chapter \ref{economicmodels}: \emph{Economic models}, we define the demand model by assuming that customers seek trips with
small travel times and the supply model by assuming that transport service providers aim to maximize profit. Given these models,
where does the demand for transportation meet the supply? How do regulation policies affect the economic equilibrium?

% \subsection{Vehicle routing}
% 
% 
% In order to provide efficient stop-to-stop service, %by vehicle mileage and trip duration, 
% it is required that the pick-up and drop-off points of customers are chosen in a way that 
% the vehicle mileage and the excess ride time reflected to customers on board is minimized while 
% the walking distance remains within reasonable limits.
% 
% \subsection{Journey planning}
% A demand-responsive transport service should be compatible with other transportation
% modes. In conventional dial-a-ride services, trip bookings are made by calling the service provider, but
% a modern service should certainly be automated and enable on-line trip booking in order
% to attract a large population. This requires a route planner that is capable of 
% communicating with conventional public transport as well as DRT.
%    
% \subsection{Economic models}


% The remainder of this document is organized as follows. In chapter \ref{potential},
% the potential for passenger pooling and trip combining is evaluated by means of
% a simplified simulation model. In chapter \ref{irdarp}, the problem of dispathcing
% vehicles in a dynamic DRT service is formalized and a solution approach is described.
% Chapter \ref{simutools} introduces a generalized simulation model designed for large scale DRT and
% a collection of simulation results for an unconstrained immediate-request service.
% An adjustable single vehicle routing algorithm, an important subroutine in the proposed 
% solution, designed for constrained problems arising in real-life situations, 
% is described in chapter \ref{eadarp}.


%% Examples of article references, remove these from your manuscript!
% Uncomment them, if you want to see the results of these commands in this example document

 % Refer to the Journal paper 1 of this example document
%\citeub{j1} \& \cpub{j1} \& \cp{j1} \& \pageref{j1} \& \ref{j1}

% Refer to the Conference paper of this example document
%\citeub[p.~2]{c1} \& \cpub[Sec.~ 1]{c1} \&  \cp[pp.~1--2]{c1} \& \pageref{c1} \& \ref{c1} 

\chapter{Vehicle Routing}
\label{vehiclerouting}
A vast majority of theoretical studies related to demand-responsive transport are formalized as combinatorial
optimization problems involving the construction of vehicle routes with respect to
a set of customers, whose pick-up and drop-off points are known a priori \cite{toth03}.
This problem is often referred to as the \emph{dial-a-ride problem}.
A large scale system operating in real time induces new challenges: 
In order to be able to compete with private cars, service
should be available within a short period of time from the trip request.
This calls for adaptive routing algorithms, since the modifications 
in vehicle routes have to be executed in real-time.
In order to ensure a sufficient level of service, the customers' waiting and ride times 
should be relatively limited. The vehicle dispatching algorithms should be designed
in a way that the constrained nature of the problem is taken into account.

In Section \ref{singlevehicle}, we study the \emph{single-vehicle dial-a-ride problem} which involves the construction
of an optimal single vehicle route serving a set of customers. This problem is motivated by the
fact that solutions to the single-vehicle problem can be used as subroutines in environments with 
multiple vehicles \cite{psaraftis01,psaraftis02}. The extension from the single-vehicle problem to the multi-vehicle
case is dicussed in Section \ref{multivehicle}.

\section{Single-vehicle dial-a-ride problem}
\label{singlevehicle}
Different types of dial-a-ride services give rise to
different types of mechanisms for controlling vehicle operations. 
For example, if a dial-a-ride service requires customers to request 
service during the previous day, the nature of vehicle
dispatching will certainly differ from a service in which
customers may request immediate service.
In this section, a specific version of the dial-a-ride problem (DARP) is examined, namely, the \emph{single-vehicle DARP
with time windows}, in which the goal is to determine the optimal
route for a single vehicle serving a certain set of customers.

The consideration of narrow time windows means that the vehicle route is 
restricted by relatively strict time limits for pick-up and delivery of each customer.
Narrow time windows emerge in real-time dial-a-ride services, in which each customer
is given an estimate or guarantee regarding the pick-up and delivery times in the 
form of time windows. These time windows are examined as hard limits to be met by the vehicle.
Time windows have been incorporated in many early and recent studies of the DARP, see for example references 
\cite{psaraftis02, jaw, madsen, toth02,cordeau02,diana, wong, cordeau01}.
In these studies it is noted that in dynamic settings, time windows eliminate the possibility of indefinite
deferment of customers and strict time limits help provide reliable service.
While the main focus is on fixed time windows, 
maximum ride time constraints that limit the time spent by passengers on the vehicle
between their pick-up and their delivery \cite{hunsaker}, 
are considered as well.

In \cite{psaraftis01}, the objective function is defined as a
generalization of the objective function of the Traveling Salesman Problem (TSP), 
in which a weighted combination
of the time needed to serve all customers and of the total degree of dissatisfaction
they experience until their arrival to the destination of the trip is minimized. The dissatisfaction of
customers is assumed to be a linear function of the time each customer waits to be picked up and the 
time each customer spends riding in the vehicle until his/her delivery.
In \cite{psaraftis02}, the approach is extended to handle time windows on departure and arrival times,
but only the route duration is minimized. 

In \ref{jeadarp}, both aspects of the problem (general objective function and time windows)
are considered. The main contribution is a solution method designed in a way that i) it is capable of
handling practically any objective function suitable for dynamic routing and ii) the computational
effort of the algorithm can be controlled smoothly: If the problem size is reasonable, 
the algorithm produces optimal solutions efficiently and as the problem size is increased,
the search space may be narrowed in order to achieve locally optimal solutions.

Publications \cp{jhitsdarpts} and \cp{jrbrorl} discuss a single-vehicle algorithm based on hyperlink-induced topic search \cite{kleinberg} 
for maximizing the number of customers in a single vehicle route. The method is seen to
be useful in determining the feasibility of multi-vehicle instances. 

\subsection*{Literature review}
In the following paragraphs, a short review on the studies related to the single vehicle dial-a-ride problem is
presented, based on \cite{cordeau03, cordeau05, cordeau07} and \cite{berbeglia}, to which the reader is referred 
for more exhaustive summaries. 

The first exact approach to the solution 
of the dynamic dial-a-ride problem was presented by \cite{psaraftis01}. 
In this work the single-vehicle, "immediate-request" case is studied, in which a set of customers should be served as soon
as possible. In this first model no time windows are specified by the customers, but the
vehicle operator incorporates \emph{maximum position shift} constraints limiting the difference between
the position of a request in the chronological calling list and its position in the vehicle route. 
The objective is to minimize a weighted combination of the time needed to serve all customers 
and customer dissatisfaction.
The complexity of this algorithm is $\mathcal{O} (n^2 3^n)$ and only small instances
can thus be solved. However, the computational effort is decreased as more strict constraints
are imposed. 

The algorithm is first constructed for the static case of the problem and then 
extended to solve the dynamic case, in which 
new requests occur during the execution of the route but no information on future requests is available.
In this version of the problem, the use of maximum position shift constraints is essential, in order 
to prevent a request from being indefinitely deferred. 
In a later study \cite{psaraftis02},
the approach is extended to handle time windows on departure and arrival times.
In this extension, only the route duration is considered in the objective
function. 

In \cite{sexton02, sexton03}, a heuristic approach to the single vehicle DARP
with one-sided time windows was introduced. 
In this problem formulation, the objective function is defined as a function of (i) the difference between the actual
travel time and the direct travel time of a user and (ii) the difference
between desired drop-off time and actual drop-off time. The algorithm was
tested on real-life problems involving 7 to 20 customers. %data sets where the number of customers varies between 7 and 20.

In \cite{desrosiers01}, an exact dynamic programming algorithm for the single vehicle DARP, formulated as an integer program, 
was presented. The formulation includes time windows as well as vehicle capacity 
and precedence constraints. Optimal solutions with respect to total route length were obtained for $n=40$.

In \cite{bianco}, exact and heuristic procedures 
for the traveling salesman problem with precedence constraints are introduced, based on
a bounding procedure. Computational results of the algorithm are given for a number
of randomly generated test problems, including the dial-a-ride problem with the classical
TSP objective function.

Psaraftis noted that \emph{although single-vehicle Dial-A-Ride systems do not exist in 
practice, single-vehicle Dial-A-Ride algorithms can be used as subroutines in
large scale multivehicle Dial-A-Ride environments. It is mainly for this reason
that one's ability to solve the single-vehicle DARP is considered important.}
A similar approach is used in this work, where the idea is to use the
solution of the single-vehicle DARP as a subroutine in a dynamic multiple-vehicle
scenario. An effort is made to solve the single vehicle problem up to optimality
whenever the problem size is reasonable. 

\subsection{Problem formulation \cite{psaraftis01,psaraftis02}}
\label{staticnarrow}
Let there be $N$ customers, each of which have been assigned a number $i$ between 1 and $N$.
For each customer $i \in \{1,\ldots,N\}$, let $u_i$ be the pick-up point and $u_{N+i}$ be the delivery point,
$[e_i,l_i]$ be the pick-up time window and $[e_{N+i},l_{N+i}]$ be the delivery time 
window\footnote{In addition to the fixed time windows $[e_i,l_i]$ and $[e_{N+i},l_{N+i}]$, to each 
customer may be associated a specific maximum ride time constraint $T_i^{\max}$. 
The handling of these constraints is discussed in section \ref{considerations}.}, 
$q_i$ be the load (number of passengers) associated to customer $i$. 
Let $A$ be the starting point of the vehicle (location of the vehicle at $t=0$).
It is assumed that the time to go from any one of the pick-up and delivery points 
$u_i$, where $i \in \{1, \ldots, 2N\}$, directly to another point $u_{j}$ is a 
known and fixed quantity $t(i,j)$.

The goal is to find a vehicle route starting from $A$ and ending at one of
the delivery points so that the following conditions hold.
\begin{enumerate}
\item
The quantity
\begin{align}
\label{tavoitefunktio01}
w_1 T + w_2 \sum_{i=1}^N (\alpha T_i^W + (1-\alpha) T_i^R)
\end{align}
is minimized, where
$w_1,w_2 \in \mathbb{R}$ are given weight parameters,
the parameter $\alpha$ is the customers' time preference constant $(0 \leq \alpha \leq 1)$,
$T$ is the duration of the route,
$T_i^W$ is the waiting time of customer $i$, from the earliest pick-up time until the time of pick-up,
$T_i^R$ is the riding time of customer $i$, from the time of pick-up until the time of delivery.
\item
The vehicle route should be legitimate, namely each customer should be picked
up before he/she is delivered.
\item
The vehicle has a certain capacity of $C$ passengers that cannot be exceeded.
\item
All time constraints must be satisfied.
\end{enumerate}

In equation \eqref{tavoitefunktio01}, the quantity $\sum_{i=1}^N (\alpha T_i^W + (1-\alpha) T_i^R)$ is the assumed
form of the total degree of dissatisfaction experienced by the customers
until their delivery, where $\alpha$ is the customers' time preference
constant describing the impact of ride time and waiting time on the degree of
dissatisfaction. 

In reference \cite{powell95}, it is noted that in static and deterministic transportation models, finding an 
appropriate objective function is fairly easy and that the objective function is 
usually a good measure for evaluating the solution. In dynamic models, the objective function
used to find the solution over a rolling horizon has often little to do with
the measures developed to evaluate the overall quality of a solution.
In stochastic models it might be useful to minimize the probability of violating time windows.
However, the advanced insertion method to be described is designed in a way that several objective 
functions, that are thought to be suitable
for dynamic and stochastic problems, may be incorporated with minimal work.
Generally, the choice of the objective function depends 
on the particular version of the problem and is
discussed only briefly in this document. 
For a study related to choosing an appropriate objetive function
for the dynamic DARP, the reader is referred to \cite{hyytia}.


The feasibility of time windows is governed by the following two assumptions.
\begin{enumerate}
	\item 
	If the vehicle arrives at any node (either pick-up or delivery)
	later than the upper bound of that node's time constraint ($l_j$),
	then the entire vehicle route and schedule is infeasible. In other words,
	those upper bounds constitute hard constraints that should be met
	by the vehicle. 
	\item
	If the vehicle arrives at any node earlier
	than the lower bound on that node's time constraint ($e_j$),
	then the vehicle will stay idle at that point and depart immediately 
	at $e_j$. 
\end{enumerate}

\subsection{Adaptive insertion algorithm [\citepub{jeadarp}]}
In general, exact procedures for solving routing  
problems are computationally very taxing, since the complexity is always more or 
less equal to the classical traveling salesman problem.
In addition, exact optimization can be seen to be needless at run-time if routes are modified often. 
Despite these facts, exact algorithms are useful in a way that the 
performance of different heuristics may be compared to the optimal solution. 

The main idea in the algorithm presented in \citepub{jeadarp} is that customers are added to 
the vehicle route one by one by using an \emph{exhaustive insertion} method,
which leads to a globally optimal solution, that is a vehicle route 
which is feasible with respect to all customers and minimizes a given cost function. 

Many studies related to the dial-a-ride problem, see for example \cite{jaw,madsen,diana,wong},
make use of what is called the insertion procedure, in the classical version of which   
the pick-up and delivery node of a new customer are inserted into the 
\emph{current optimal sequence} of pick-up and delivery nodes of existing customers.
In these references, several improvements to the insertion algorithm 
have been suggested. However, the main idea
of the classical insertion algorithm a priori
excludes the possibility that the appearance of a new customer may render
the optimal sequencing of already existing customers no longer optimal.
While the basic insertion algorithm can be seen to
produce relatively good results for unconstrained problems, 
the performance compared to an exact algorithm is decreased as the
problem becomes more constrained.

The main idea is to construct the optimal route iteratively by implementing 
an insertion algorithm for each customer, one by one \emph{for all feasible sequences 
of pick-up and delivery nodes of existing customers}.
Namely, the procedure involves two steps for each customer:
\begin{enumerate}
	\item 
	Perform insertion of the new customer to all feasible service sequences with respect to
	existing customers.
	\item
	Determine the set of feasible service sequences with respect to the new customer and existing customers.
\end{enumerate}

It can be readily shown that the insertion of a new customer to all feasible service sequences with respect to 
existing customers 
produces all feasible service sequences with respect to the union of existing customers 
and the new customer and leads to a globally optimal solution but is computationally expensive
if the number of feasible service sequences grows large. However, if the route is constructed under relatively narrow time window constraints,
the number of feasible routes with respect to all customers
will be small compared to the number of all legitimate routes. 
Furthermore, the algorithm may be easily extended to an adjustable heuristic algorithm
able to handle any types of time windows, see Section \ref{structure}. 

The idea of the advanced insertion method is 
clarified by the following example, where no
capacity or time constraints are taken into account. 
Let $i^{\uparrow} = i$ denote the \emph{pick-up node} of customer $i$ and let $i^{\downarrow} = N + i$ denote 
the \emph{delivery node} of customer $i$.
A \emph{service sequence} is defined as an ordered list consisting of pick-up and delivery nodes.
For instance, the service sequence $( i^{\uparrow},  j^{\uparrow}, j^{\downarrow}, i^{\downarrow})$ indicates the order
in which customers $i$ and $j$ are picked up and dropped off.

Let us start the advanced insertion process with customer 1.
Since the pick-up $1^{\uparrow}$ of customer 1 has to be before the delivery, $1^{\downarrow}$, 
the only possible service sequence at this point is $( 1^{\uparrow},  1^{\downarrow})$. Thus, the 
set of potential service sequences with respect to customer $1$ consists of this single service sequence. 
By insertion of customer $2$ into the service sequence $( 1^{\uparrow}, 1^{\downarrow})$ we get the six 
service sequences presented in table \ref{ykskakstaulukko01}.
\begin{table}[ht] 
\caption{Potential service sequences with respect to customers 1 and 2. 
No capacity or time constraints are taken into account. 
$i^{\uparrow}$ denotes the pick-up node and $i^{\downarrow}$ denotes
the delivery node of customer $i$.} 
\centering     
\begin{tabular}{|rrrrrr|rrrrrr|rrrrr|}  
\hline  & & & & & & & & & & & &  & & & &  \\ [-0.7em]                
A: & $ 1^{\uparrow} $ & $  1^{\downarrow} $ & $  2^{\uparrow} $ & $  2^{\downarrow} $ & & 
B: & $ 1^{\uparrow} $ & $  2^{\uparrow} $ & $  1^{\downarrow} $ & $  2^{\downarrow} $ & & 
C: & $ 1^{\uparrow} $ & $  2^{\uparrow} $ & $  2^{\downarrow} $ & $  1^{\downarrow} $ \\ [1ex]
\hline & & & & & & & & & & & &  & & & &  \\ [-0.7em]
D: & $ 2^{\uparrow} $ & $  1^{\uparrow} $ & $  1^{\downarrow} $ & $  2^{\downarrow} $ & & 
E: & $ 2^{\uparrow} $ & $  1^{\uparrow} $ & $ 2^{\downarrow} $ & $  1^{\downarrow} $ & & 
F: & $ 2^{\uparrow} $ & $  2^{\downarrow} $ & $  1^{\uparrow} $ & $  1^{\downarrow} $ \\ [1ex] 
\hline                      
\end{tabular} 
\label{ykskakstaulukko01} 
\end{table} 


By inserting the pick-up and delivery node of customer $3$ into all of these service sequences we get a total
of $6(5+4+3+2+1) = 90$ new potential service sequences. However, if the time and capacity
constraints are taken into account, not all service sequences described above are necessarily feasible.

For example, if after the insertion of customer $2$ it can be seen that only the
service sequences A and B in table \ref{ykskakstaulukko01}, namely 
$( 1^{\uparrow}, 1^{\downarrow}, 2^{\uparrow}, 2^{\downarrow})$ and $( 1^{\uparrow}, 2^{\uparrow}, 1^{\downarrow}, 2^{\downarrow})$,
are feasible with respect to time constraints, then it can be
shown that all feasible service sequences including customer $3$ are included in
the sequences given by insertion to these two service sequences, which gives
a maximum of $2(5+4+3+2+1) = 30$ new potential service sequences.

Briefly, the structure of the advanced insertion algorithm may be sketched as follows.
At first, the set of feasible service sequences $S_i$ is determined recursively
for each customer $i \in \{1, \ldots, N\}$ by inserting the pick-up and delivery nodes of customer $i$ into each feasible
service sequence with respect to customers $1,\ldots,i-1$. In this way the algorithm 
produces the set $S_N$ of all feasible routes with respect to customers $1,\ldots,N$. 
Then the solution to the static problem is obtained by choosing the sequence $s \in S_N$ with
minimal cost $C(s)$. In the general form \eqref{tavoitefunktio01} of the cost function, this involves
the calculation of waiting times and ride times for all customers
for all feasible service sequences. If only route duration is minimized,
we can simply choose the sequence for which the arrival time at the last
node is smallest.



\subsubsection{A priori clustering}
\label{clustering}
In problems involving a large number of customers, assuming that the time windows are relatively narrow,
a significant portion of service sequences can be eliminated before 
the actual insertion process by simply studying the mutual relationships between nodes, similarly 
as described in \cite{dumas03}.
More precisely, assume that the vehicle departs from a node $i$ at the lower
bound $e_i$ of the time window and moves directly to node $j$. If the vehicle 
does make it in time to $j$, that is, if
	\begin{align}
		e_{i} + t(i,j) & > l_{j},
	\end{align}
it is said that the transition $i\to j$ is \emph{a priori infeasible}.
A priori infeasibility could be defined similarly for capacity constraints.
However, assuming that the load associated to each customer 
is at most $C/2$, where $C$ is the capacity of the vehicle,
each transition is a priori feasible with respect to capacity.
In other words, two customers $i$ and $j$ with loads satisfying $q_i \leq C/2$ and $q_j \leq C/2$
may be picked up and dropped of in any order without the capacity constraint 
being violated.


Note that if for any two nodes $i$ and $j$, both transitions $i \to j$ and 
$j \to i$ are infeasible a priori, the entire problem is infeasible. 
Otherwise, either $i \to j$ or $j \to i$ or both are a priori feasible and the pick-up and delivery nodes of customers
can be divided among $m$ preceding clusters by using the following rule.

\begin{definition}
Cluster $C_k$ precedes cluster $C_l$ 
if and only if the transition $x \to y$ is a priori infeasible for all $x \in C_l$ and
$y \in C_k$. In this case, we shall use the notation $C_k \prec C_l$.
\end{definition}

Clearly, assuming that there exists a feasible solution, the above precedence relation 
of clusters is a strict order: (i) $C \not\prec C$ for all clusters, (ii) $C_a \prec C_b$
implies $C_b \not\prec C_a$ and (iii) $C_b \prec C_c$ and $C_a \prec C_b$ implies
$C_a \prec C_c$. In addition, note that if a single node $i$ has an infinite time window $[-\infty,+\infty]$, there is 
only one cluster since all transitions $j \to i$
and $i \to j$ are feasible a priori.

In practice, the clusters can be determined by means of an adjacency matrix $F$ for which $F_{ij}=1$ if $i\to j$ is
feasible and $F_{ij}=0$ if $i\to j$ is infeasible a priori. By arranging the rows and
columns suitably we get a block upper triangular matrix where each block corresponds to
a cluster. Figure \ref{clusterexample01} shows the adjacency matrix of a sample problem
involving four customers. The rows and columns are arranged in a descending order 
with respect to row sums. From the matrix five clusters can be identified (blocks on the diagonal), namely 
$\{1^{\uparrow} \}	\prec \{ 1^{\downarrow}, 3^{\uparrow}, 2^{\uparrow}\} \prec \{ 2^{\downarrow} \} \prec  
\{ 3^{\downarrow}, 4^{\uparrow} \} \prec  \{ 4^{\downarrow} \}$. 
Thus, since the positions of nodes $1^{\uparrow}, 2^{\downarrow}$ and $4^{\downarrow}$ are fixed,
the problem falls down to determining the 
optimal ordering of nodes $1^{\downarrow}, 3^{\uparrow}, 2^{\uparrow}$ and 
$3^{\downarrow}, 4^{\uparrow}$. 
\begin{figure}[ht]
\begin{center}
\begin{align*}
\newcommand*{\temp}{\multicolumn{1}{c|}{0}}
\newcommand*{\tempi}{\multicolumn{1}{c|}{1}}
\begin{array}{c|cccccccc}
   & 	1^{\uparrow}	   &  1^{\downarrow}   &  3^{\uparrow}   &  2^{\uparrow}   &  2^{\downarrow}   &  3^{\downarrow}   &  4^{\uparrow}   &  4^{\downarrow} \\
   \hline
1^{\uparrow} &     1   &  1   &  1   &  1   &  1   &  1   &  1   &  1 \\ \cline{2-2}
1^{\downarrow} &     \temp   &  1   &  1   &  1   &  1   &  1   &  1   &  1 \\
3^{\uparrow} &     \temp   &  1   &  1   &  1   &  1   &  1   &  1   &  1 \\
2^{\uparrow} &     \temp   &  1   &  1   &  1   &  1   &  1   &  1   &  1 \\ \cline{2-5}
2^{\downarrow} &     0   &  0   &  0   &  \temp   &  1   &  1   &  1   &  1 \\ \cline{2-6}
3^{\downarrow} &     0   &  0   &  0   &  0   &  \temp   &  1   &  1   &  1 \\
4^{\uparrow} &     0   &  0   &  0   &  0   &  \temp   &  1   &  1   &  1 \\ \cline{2-8}
4^{\downarrow} &     0   &  0   &  0   &  0   &  0   &  0   &  \temp   &  1 \\ \cline{2-8}
\end{array}
\end{align*}
\caption{A priori adjacency matrix of a sample problem involving $4$ customers. From the matrix five clusters can be identified, namely 
$\{1^{\uparrow} \}	\prec \{ 1^{\downarrow}, 3^{\uparrow}, 2^{\uparrow}\} \prec \{ 2^{\downarrow} \} \prec  
\{ 3^{\downarrow}, 4^{\uparrow} \} \prec  \{ 4^{\downarrow} \}$.}
\label{clusterexample01}
\end{center}
\end{figure}
In general, by clustering the nodes a priori, a significant amount
of insertions need not be checked for feasibility. 


\subsubsection{Structure and complexity}
The number of insertions that are tested for feasibility for customer $i$ is bounded above 
by the formula
\begin{align}
n(i) \leq m_{i-1} \sum_{x=1}^{2i-1} x  = \frac{m_{i-1}(2i)(2i-1)}{2},
\end{align}
where $m_{i-1}$ is the number of feasible service sequences with respect to customers $1, \ldots,i-1$.
The inequality can be justified by the fact that not all possible insertions  
are checked for feasibility. 
In the worst case, where capacity and time constraints are \emph{not restrictive},
we get 
\begin{align*}
& m_1 = 1, \ \ \ \ \ \ \ m_2 = (4 \cdot 3)/2 = 6,  \ \ \ \ \ \ \ m_3 = 6(6 \cdot 5)/2 = 90, \\
& m_i = \frac{(2i)(2i-1)}{2} \frac{(2(i-1))(2(i-2)-1)}{2} \cdots \frac{4 \cdot 3}{2} = \frac{(2i)!}{2^i}. 
\end{align*}
Thus the maximum number of insertions required in the solution of the static case is
$\sum_{i=1}^{N}\frac{(2i)!}{2^i} \approx \sum_{i=1}^{N} {2^{1-i} i^{2i + \frac 12} \sqrt{\pi}}$.
Thus, in the worst case, the number of feasible solutions is of order
$\mathcal{O}(\sqrt{N} (N^2/2)^N)$.

Suppose that, due to strict time (and capacity) constraints, the number of potential service sequences $m_i$ for customers $1,\ldots, i$
is bounded by some function $m:\mathbb{N} \to \mathbb{N}$ such that $m(i) \leq \frac{(2i)!}{2^i}$.
Then the number of insertions for customer $i$ is bounded by $n(i) \leq \frac{m(i-1)(2i)(2i-1)}{2}$.
The total number of insertions is bounded by $\sum_{i=1}^{N}\frac{m(i-1)(2i)(2i-1)}{2}$.
For example, if $m(i) = kN$ for some constant $k$, the computational complexity of the screening phase
is reduced to $\mathcal{O}(N^4)$. 
If $m(i) = k$, the computational complexity is reduced to $\mathcal{O}(N^3)$.

On the grounds of previous calculations, it can be stated that the adaptive insertion
algorithm will generally not be able to produce exact solutions efficiently in cases where
the capacity and time constraints are not restrictive. However, if the number
of feasible service sequences is bounded due to strict constraints, the 
algorithm will lead to an exact solution computationally inexpensevely.
In addition, the advanced insertion 
algorithm has a special property of being extendable to an adjustable heuristic,
as described in the following subsection.

\subsubsection{A heuristic extension}
\label{heuristic}
Even if the capacity and time constraints were not highly restrictive, the algorithm can be 
modified easily by bounding the size of the set $S_i$ 
of service sequences, in which new customers are inserted, 
by including only a maximum of $L$ service sequences
for each customer $i$. More precisely, if after inserting 
customer $i$, the number of feasible service sequences with respect to customers $1,\ldots, i$
is larger than $L$, the set of feasible service sequences $S_i$ with respect to 
customers $1,\ldots, i$ is narrowed by including only $L$  
service sequences, that seem to allow the insertion of remaining customers (see part \ref{hobjfunc}).
After the last customer has been inserted, the feasible service sequences are evaluated
by means of the objective function \eqref{tavoitefunktio01}.

This modification leads to a heuristic algorithm, in which the computational
effort can be controlled by the parameter $L$, referred to as the \emph{degree} of
the heuristic. The resulting algorithm is
somewhat sophisticated in a way that it produces globally optimal solutions for
small sets of customers and when the number of customers is increased, the
algorithm still produces locally optimal solutions with 
reasonable computational effort. In the special case where $L=1$, the
algorithm reduces to the classical insertion algorithm. 
If $L \geq \frac{(2N)!}{2^N}$, the heuristic coincides with the exact version of the 
algorithm as no routes are discarded.

\subsubsection{Objective functions}
\label{hobjfunc}
In order to be able to efficiently make use of the above heuristic extension
idea, the set of service sequences is narrowed by means of a certain heuristic objective function
after the insertion of each customer.
Since the main purpose of heuristics at the operational level is to always produce some implementable 
solutions very quickly, even if they were only locally optimal, such an objective function should 
be defined in a way that the algorithm is 
capable of producing \emph{feasible} solutions even if the complexity of the problem was high. %was highly constrained.

Looking only at the cost defined in formula \eqref{tavoitefunktio01} may eliminate from consideration
sequences that are marginally costlier but would easily allow the
insertion of remaining customers in the route. 
Thus, more sophisticated criteria should be considered to help ensure that the heuristic will
find a feasible solution when one exists.

In other words, the function should favor service sequences with enough 
\emph{time slack} for those customers, that have not been inserted into the sequences.
\ref{jeadarp} suggests the following heuristic objectives.
Given the service sequence $s = (p_1, \ldots, p_{m})$, we wish to optimize one 
of the functions 
\begin{align}
\label{rl}
f_{rl}(s) &= t_m, & \mbox{(Route duration)} \\
\label{ts}
f_{ts}(s) & = \sum_{j=1}^{m} l_j -t_j ,  & \mbox{(Total time slack)} \\
\label{maxmin}
f_{min}(s) &= \min_{j \in \{1, \ldots, m\}}  l_j - t_j, & \mbox{(Max-min time slack)} 
%\label{lastnode}
%f_{LN}(s) &= l_m - t_m, & \mbox{(Last node time slack)} 
\end{align}
where $[e_j,l_j]$ is the time window and $t_j$ is the calculated time of arrival at node $p_j$.

In general, each of the above objective functions aim to 
maximize the temporal flexibility of service sequences in different ways.
\begin{description}
\item[Route duration]
%The route duration objective function 
\eqref{rl} favors service sequences in
which the time to serve all customers is as small as possible. 
This objective can be justified by the fact that 
it is likely that new customers may be inserted at the rear of 
a route that is executed quickly.
\item[Total time slack] 
\eqref{ts} stores sequences in which
the sum of excess times (or the average excess time) at the nodes is maximized, that
is, sequences which are likely to allow the insertion of a new customer %can be inserted
before the last node.
\item[Max-min] \eqref{maxmin} seeks sequences in which the 
minimum excess time at the nodes of the route is maximized. In other words,
the sequences in which there is at least some time slack at each node
are considered potential.
\end{description}

A simple example motivating the use of the above objective functions is 
presented in figure \ref{flexibility01}.

\begin{figure}[ht]
\begin{center}
\includegraphics[width=0.65\textwidth]{flexibility03.pdf}
\caption{Route flexibility. A vehicle is located at $A$ at $t = 0$,
and two customers are due to be picked up within the presented time windows at $i$ and $j$.
The dashed lines represent two possible routes for the vehicle.
If the route duration were minimized, $i$ should be visited before $j$. However, since
there would be no "slack time" at $j$, this decision would a priori 
exclude the possibility that new customers could be inserted between $i$ and $j$. 
On the other hand, if $j$ were visited before $i$, there would be
more possibilities for inserting new customers on the route before $i$. However, 
the route $A \to i \to j$ is shorter and thus it is more likely that 
customers can be inserted at the end of the sequence.}
\label{flexibility01}
\end{center}
\end{figure}



\subsubsection{Tuning}% $K$-multiplying}
\label{doubling}
In order to ensure that the heuristic always produces a feasible solution when one exists,
the parameter $L$ can be tuned during run-time by using the following idea. 
\begin{enumerate}
\item
At first, the problem is solved by using an initial degree $L_0$. 
\item
Each time the algorithm is unable to find a feasible solution, the degree is increased and 
the problem is solved again.
\end{enumerate}
Let us study the expected CPU time of the tuning method.
Let $P(S|L=l)$ denote the probability that a solution is found by using the degree $L=l$
and let $T(l)$ denote the average running time of the algorithm with $L=l$. 
The average CPU times of feasible runs are assumed equal to those occurring during infeasible runs, although
infeasible runs are actually slightly less expensive. However, it can be seen that this approximation 
will not affect the results of the following examination. 
The expected CPU time for a given tuning strategy $(L_0,L_1,\ldots)$, where $L_0 < L_1 < L_2 < \ldots$, 
is given by the formula %can be written in the form
\begin{align*}
%E(\tau(\textbf{L})) & = 
T(L_0) + \sum_{i=1}^{\infty} T(L_i) \prod_{j=0}^{i-1} (1- P(S|L=L_i)). 
\end{align*}
It should be noted that the above probabilities are actually conditional: If a solution
is not found with some value of $L$, there is no point in solving the problem 
again with the same value. In addition, a relatively small increase in 
the value of $L$ will not significantly affect the possibilities
of finding a solution. Thus, the tuning approach used in the following examination
is based on multiplying the value of $L$ by a number $K$ each time a feasible
solution is not found. This method will be referred to as $K$-multiplying.

Assuming that %for sufficiently small values of $l$, 
the complexity grows linearly with $l$, that is $T(l) = l T(1)$, and %, which implies that $E(\tau(L_0)) \geq T(L_0)$. 
that the probability of finding a solution is sufficiently high, %$1/2 < P(S|L=1) < P(S|L=2) < \ldots$, 
it can be shown that the optimal initial degree
for the $K$-multiplying method has to be relatively small. %is $L_0=1$. 
At first we will show %, for the $K$-multiplying method, where $K \geq 2$ is a natural number, 
that if $P(S|L \geq 1)>1/2$, then %and $k=2$ corresponds to the doubling method, 
any initial degree $L_0 = mK$, where $m \in \mathbb{N}$, is suboptimal. %the optimal value satisfies $1 \leq L_0 < k$,

\begin{theorem}
\label{doublingthm01}
Let $E(\tau(L_0))$ denote the expected CPU time of the $K$-multiplying method for the initial value $L_0$. If
$T(l) = l T(1)$ for $l \in \mathbb{N}$ and $P(S|L \geq 1) > 1/2$, then
\begin{align*}
E(\tau(L_0)) \leq E(\tau(KL_0))
\end{align*}
for all $L_0 \in \mathbb{N}$ and $K \geq 2$.
\end{theorem}
\begin{proof}
Since 
it is clear that $E(\tau(l)) \geq T(l)$ for $l \in \mathbb{N}$, for any $K \geq 2$, we get 
\begin{align*}
E(\tau(L_0)) & = T(L_0) + (1-P(S|L=L_0)) E(\tau(KL_0)) < T(L_0) + \frac 12 E(\tau(KL_0)) \\
								& = \frac 1k T(KL_0) + \frac 12 E(\tau(KL_0)) \leq (\frac 1K + \frac 12) E(\tau(KL_0)) \leq E(\tau(KL_0)).
\end{align*}
\end{proof}

Then, the following theorem states that if the probability of finding a solution
is at least $1 - 1/K$, the optimal initial value is 
bounded above by the logarithm of the smallest natural number $l$ for which $P(S|L=l)=1$.

\begin{theorem}
\label{doublingthm02}
Let $l'$ denote the smallest natural number $l$ for which $P(S|L=l)=1$.
If $P(S|L \geq 1) > 1 - 1/K$, then 
\begin{align*}
E(\tau(L_0)) < T(L_0(1+\log_K l'/L_0))
\end{align*}
for all $L_0 \in \mathbb{N}$ and for all $K > 1$.
\end{theorem}
\begin{proof}
The expected CPU time is given by the formula
\begin{align*}
E(\tau(L_0)) &= \sum_{i=0}^{\infty} (1-P(S|L=K^iL_0))^i T(K^iL_0). 
\end{align*}
Since $(1-P(S|L=K^iL_0)) = 0$ for $K^iL_0 > l'$, that is, $i > \log_K l'/L_0$, we get
\begin{align*}
E(\tau(L_0)) &= \sum_{i=0}^{\lfloor \log_K l'/L_0 \rfloor} (1-P(S|L=K^iL_0))^i T(K^iL_0) 
								 \leq \sum_{i=0}^{\lfloor \log_K l'/L_0 \rfloor} (1-(1-1/K))^{i} K^i T(L_0) \\
								 & \leq T(L_0)(1+\log_K l'/L_0).
\end{align*}
\end{proof}
In general, the two theorems imply that the expected CPU time is minimized 
when the initial degree $L_0$ is small. However, this is true only if %due to 
the probability of finding a solution with small values of $L$ is relatively high.
For example, if $l' = 16$ , $K=2$ and $P(S|L \geq 1) > 0.5$, theorem \ref{doublingthm02} states that
that $E(\tau(1)) < T(5)$. By theorem \ref{doublingthm01} we know that
the values $2$ and $4$ are suboptimal for $K=2$. Thus, the optimal initial value
is either $1$ or $3$ depending on the actual probabilities of finding solutions with different values of $L$. 
%The fact that the optimal value of $L_0$ (with respect to complexity) is small 
%is verified by the computational results presented in section \ref{experience}.

Finally, it should be noted that the optimal choice
of $K$ and $L_0$ will in general depend on the type of the problem to be solved and that
by increasing the values, the optimality of the %probability of achieving the globally optimal
solution with respect to the objective of the problem is increased as well since more sequences are evaluated,
as will be seen in section \ref{experience}. 
%In this work, however, the heuristic algorithm is designed to solve the problem as quickly as possible 
%and thus the choice of $K$ and $L_0$ which minimizes the CPU time is considered optimal.

%Since the expected CPU time is an increasing function of the initial value $L_0$,
%the value $L_0=1$ is used in the following experiments.




\subsection{Maximum cluster algorithm [\citepub{jhitsdarpts}]}






\subsection{The dynamic case}
\label{dynamicdarp}
Let us discuss the extension of the predescribed single-vehicle algorithms to 
the dynamic case, in which new customers are appended to the route in real time.
Generally, it can be seen that there is no need to compute a new route
except when a new customer request occurs, a customer does not show up at the agreed pick-up location,
the vehicle is ahead or behind of schedule or the vehicle has lost its way. 

A simple example, similar to the one presented in \cite{psaraftis01},
on the real-time updating of a vehicle route %might operate 
is shown in figure \ref{dynamicexample02}.
Initially, customers $1$ and $2$ have been assigned 
pick-up and delivery points and corresponding time windows. The vehicle located at $A$ follows the
tentative optimal route $(1^{\uparrow},2^{\uparrow},1^{\downarrow},2^{\downarrow})$,
where "$\uparrow$"-symbols denote pick-up nodes and "$\downarrow$"-symbols denote delivery nodes. 
(see figure \ref{dynamicexample02}a).
At the time the vehicle is at $B$, the vehicle route is updated with respect 
to customers $1,2$ and $3$. At this instant %in time, %the procedure revises the
% nonexecuted portion of the route and produces the 
a new tentative optimal route shown in figure \ref{dynamicexample02}b is produced. 

\begin{figure}[ht]
\begin{center}
\includegraphics[width=1.0\textwidth]{dynamicexample04.pdf}
\caption{Modifications in the vehicle route. The route is updated when the vehicle is at $B$ and 
a new tentative optimal route beginning at $B'$ is produced. 
The figures on the right show the routes as so-called labeled Dyck 
paths \cite{cori}, in which each pick-up $i^{\uparrow}$ precedes the correspoding drop-off $i^{\downarrow}$. 
At the time a new customer is added to the route, a new path is formed. Clearly, the "height" of the path
shows the number of customers aboard in different parts of the route.}
\label{dynamicexample02}
\end{center}
\end{figure}

Note that the new route does not have $B$ as origin, but
a point $B'$, slightly ahead of $B$. This is due to two facts:
i) It will take some time for the algorithm to process the new input and reoptimize, 
ii) It will take some time for the driver
of the vehicle to process the information regarding the new route. 
The distance between $B$ and $B'$ depends in general on the particular 
dial-a-ride service examined. For example, in a highly dynamic setting,
in which the modification of the route is not
allowed to take more than a few seconds, it may be assumed that
the latter of the facts is more restrictive. 
Any dynamic dial-a-ride
service should make use of a mechanism to update the point $B'$
at sufficient time intervals.
In this work, however, it is assumed that the point $B'$ is
known at each instant, as well as the estimated time of arrival $T'$ at $B'$.
The \emph{vehicle checkpoint} $(B',T')$ acts as the starting point for all service sequences.

	For customers that have already been picked up, the pick-up point 
	is not a part of the input of the problem. Thus, for such customers
	only the delivery point is considered.

	In dynamic models, the objective function
	used to find the solution over a rolling horizon has often little to do with
	the measures developed to evaluate the overall quality of a solution \cite{powell95}.
	However, the adaptive insertion method described in \ref{jeadarp} is designed in a way that several objective 
	functions, that are thought to be suitable
	for the dynamic problem, may be incorporated with minimal work.
	For example, if the vehicle route is subject to several modifications
	in short periods of time, one of the flexibility measures \eqref{maxmin}, \eqref{ts} 
	or forward time slack defined in \cite{savelsbergh} may be used as an objective 
	function for the problem in order to be able to insert as many future customers in the 
	route as possible. Generally, the choice of the objective function depends 
	on the particular version of the dynamic routing problem and the performance
	of different objective functions may be sensitive to, for example, constraints, demand intensity
	or the number of vehicles. For a study related to choosing an objetive function
	for the immediate-request DARP, the reader is referred to \cite{hyytia}.


\section{Numerical experiments}
\label{experience}
The following paragraphs present a summary of the computational results reported in \ref{jeadarp}. The exact and heuristic versions of the algorithm
were tested on a set of problems involving different numbers of customers and different time
window widths determined by a travel time ratio $R$ describing the maximum allowed ratio of travel time
to direct ride time. The pick-up and drop-off points of customers were chosen randomly from a
square-shaped service area and the ride times between the points were modeled by euclidean distances.

At first, the complexity of the problem was studied with respect to three parameters, namely
i) the number of customers $N$, ii) travel time ratio $R$ and iii) the average time interval $\mu$
between customer requests. The complexity of the exact algorithm was measured in terms of the
average number of feasible sequences
in \emph{feasible problem instances}, that is, randomized
problems for which at least one feasible solution was found. 
The number of feasible sequences gives us an insight on the complexity of the problem itself,
since any enumeration algorithm would have to evaluate the same number of feasible sequences.

Then, the performance of the heuristic with different objective functions was evaluated.
The complexity was measured in terms of the number of sequences evaluated by the heuristic.


\subsection{Experiments}
The following experiments were performed on a standard laptop computer with a 2.2 GHz processor. 
The CPU times and the number of evaluated sequences appeared to have a roughly linear relationship.
A typical problem instance involving 20 customers could be solved up to optimality within less than a second.

\begin{figure}[ht]
\begin{center}
\includegraphics[width=0.7\textwidth]{nvertailu01.pdf}
\caption{Experiment 1. The complexity of the exact algorithm with respect to the number of customers on a logarithmic scale. 
The three curves represent, as a function of the number of customers, the average number of feasible sequences 
for $R=2,2.5,3$ and $\mu=1800$s. Clearly, the complexity increases exponentially with respect
to the number of customers.}
\label{nvertailu01}
\end{center}
\end{figure}



\begin{figure}[ht]
\begin{center}
\includegraphics[width=0.7\textwidth]{ttivertailu01.pdf}
\caption{Experiment 2. The complexity of the exact algorithm with respect to travel time ratio on a logarithmic scale. 
The three curves represent, as a function of the travel time ratio, the average number of feasible sequences
for $N=5,10,20$ and $\mu = 1800$s.}
\label{ttivertailu01}
\end{center}
\end{figure}


\begin{figure}[ht]
\begin{center}
\includegraphics[width=0.8\textwidth]{intvertailu01.pdf}
\caption{Experiment 3. The complexity of the exact algorithm with respect to the average time 
interval between customers. The solid lines represent the average number of feasible sequences in
feasible problem instances and all problem instances, on a logarithmic scale for $N=10$ and $R=3$.
The dashed line corresponds to the fraction
of problem instances, for which at least one feasible solution was found.}
\label{intvertailu01}
\end{center}
\end{figure}


\begin{figure}[ht]
\begin{center}
\includegraphics[width=0.7\textwidth]{objvertailu01.pdf}
\caption{Experiment 4. The performance of three different heuristic objective functions as functions of degree $L$.
The curves represent the fractions of instances for which a feasible solution was found by the objective functions 
(compared to the exact algorithm), 
for $N=20$, $R=3$ and $\mu=1800$s. The total slack time objective function
outperforms the other two in all studied cases.
}
\label{objvertailu01}
\end{center}
\end{figure}




\subsection*{Experiment 1: Number of customers}
At first we study the complexity of the exact algorithm with respect to the number of customers.
In practice, the number of customers assigned to a single vehicle is governed by the
pre-order time of the dial-a-ride service: If customers may request service in
advance, the planned vehicle routes are expected to be longer than in immediate-request
services. Figure \ref{nvertailu01} shows the average number of feasible sequences in feasible problem instances 
on a logarithmic scale, computed over 10000
randomized instances for $N=1\ldots 20$, $R = 2,2.5,3$ and $\mu = 1800$s.

Referring to the figure, it can be seen that the complexity of the problem increases exponentially
with respect to the number of customers with all studied values of the travel time ratio.  
In addition, the complexity is increased with the travel time ratio. 

\subsection*{Experiment 2: Travel time ratio}
Let us study the complexity of the exact algorithm as a function of travel time ratio.
Figure \ref{ttivertailu01} shows the average number of feasible sequences in feasible problem instances 
on a logarithmic scale, computed over 10000
randomized instances for $R=1.5\ldots 3$, $N = 5,10,20$ and $\mu = 1800$s. 


The figure shows that the effect of the travel time ratio on the complexity of the problem is significant.
The fact that the slopes of the curves increase with $R$ on the logarithmic scale indicates that the relation 
between complexity and $R$ is superexponential.


\subsection*{Experiment 3: Time interval}
Let us conclude the study of the exact algorithm by examining complexity with respect to the average time interval between customer requests. 
The solid lines in figure \ref{intvertailu01} represent the average number of feasible sequences in
i) feasible problem instances and ii) all problem instances, on a logarithmic scale, computed over 10000
randomized instances for $N=10$, $R=3$ and $\mu = 6 \ldots 40$ minutes. The dashed line corresponds to the fraction
of problem instances, for which at least one feasible solution was found.


The figure indicates that the complexity of feasible problem instances decreases exponentially 
with respect to the average time interval $\mu$. On the other hand, the probability of finding
at least one feasible solution is increased with $\mu$. By looking at the curve corresponding 
to the average complexity of all problem instances (including infeasible cases), it can be seen that the
complexity is maximized at a certain time interval ($\mu = 24$ minutes in this case), in which both the probability of
finding a feasible solution and the number of feasible sequences in feasible cases are relatively large.

At this point, it should be emphasized that the above results apply to randomized instances for the 
single vehicle problem. In a dynamic multiple vehicle setting, an effort is made to 
divide the customers among available vehicles optimally. Thus, it is suggested that a larger number of customers
can be served by a single vehicle in less time than in the case in which the customers' pick-up
and drop-off locations are completely random. However, the above results give us an insight on
how the complexity of the single vehicle subroutine behaves with respect to the average time interval between
the earliest pick-up times of customers assigned to a single vehicle.


\subsection*{Experiment 4: Objective functions}
Let us study the performance of the three different heuristic cost functions \eqref{rl}, \eqref{ts} and \eqref{maxmin}
as a function of the degree $L$ of the heuristic.
Figure \ref{objvertailu01} shows the fraction of problem instances for which a feasible solution was 
% in feasible problem instances obtained 
found by the heuristic (compared to the exact algorithm), computed over 10000
randomized instances. 
The number of customers was set to $N=20$, the travel time ratio
was $3$ and the average time interval $\mu = 1800$s was used.

Referring to the figure, it can be seen that the total time slack cost function \eqref{ts} is capable of finding
a feasible solution to randomized problems most often, while the performance of the route duration cost
function \eqref{rl} is worst of the three algorithms. Note that as the degree $L$ is increased, 
the fraction of feasible solutions converges to 1 for any heuristic cost function, since whenever $L \geq \frac{(2N)!}{2^N}$,
the heuristic coincides with the exact algorithm regardeless of the studied problem.




\subsection{Conclusions}
In this work, an exact optimization procedure is developed to solve the
static and dynamic versions of the single vehicle dial-a-ride problem with time 
windows. Using complete enumeration to solve the problem with respect
to a generalized objective function is motivated by the dynamic nature of 
online demand responsive transport services,
in which looking only at the tentative route duration, as in existing algorithms for the problem,
may decrease the possibilities of serving future customers.

In addition, an adjustable heuristic extension to the algorithm
is introduced, in order to be able to control the CPU times:
If the problem size is reasonable, the proposed solution method
produces globally optimal solutions. If the problem size is increased, 
the algorithm adjusts itself to produce locally optimal solutions,
closing the gap between the classical insertion heuristic and the exact solution and thus
making the algorithm applicable to any static or dynamic dial-a-ride problem.








\section{Multi-vehicle dial-a-ride problem}
\label{multivehicle}

\chapter{Journey planning}
\label{journeyplanning}

\chapter{Economic models}
\label{economicmodels}

\chapter{Conclusions}
\label{conclusions}
This work is focused on combinatorial problems arising from the planning of demand responsive transport.
%Particularly, the single vehicle dial-a-ride problem and related problems are studied.

% In chapter \ref{irdarp}, a decentralized algorithm for
% real-time demand responsive transport is described.
% The proposed solution makes use of communication between
% customers requesting service and vehicles providing service in a way
% particularly suitable for a highly dynamic service, in which
% a vast majority of requests are formed not long before the customer
% requesting service is willing to depart. 
% An algorithm for solving the dynamic single vehicle dial-a-ride
% problem is used as a subroutine in the multiple-vehicle problem. 
% In order to achieve high-quality results, an exact algorithm for the single vehicle dial-a-ride problem
% should be incorporated. In some applications, however, this may not be feasible since such an algorihm 
% would require too much computational work. On the other hand, the use of heuristics, such as the insertion algorithm
% (see for example \cite{jaw}) may significantly degrade the performance of the service in some cases.
% 
% In chapter \ref{eadarp}, an exact optimization procedure is developed to solve the
% static and dynamic versions of the single vehicle dial-a-ride problem with time 
% windows. Using complete enumeration to solve the problem with respect
% to a generalized objective function is motivated by the dynamic nature of 
% online demand responsive transport services,
% in which looking only at the tentative route duration, as in existing algorithms for the problem,
% may decrease the possibilities of serving future customers.
% In addition, an adjustable heuristic extension to the algorithm
% is introduced, in order to be able to control the CPU times:
% If the problem size is reasonable, the proposed solution method
% produces globally optimal solutions. If the problem size is increased, 
% the algorithm adjusts itself to produce locally optimal solutions,
% closing the gap between the classical insertion heuristic and the exact solution and thus
% making the algorithm applicable to any static or dynamic dial-a-ride problem.
% 
% Chapter \ref{tspdn} focuses on the effect of walking on the performance of
% demand responsive transport. The problem of redirecting customers is modeled by means of the traveling salesman
% problem with disk neighborhoods, in which each node can be redirected
% to another location within a certain maximum walking distance from the original location. 
% By means of differential analysis, an upper bound for the decrease 
% in the length of a vehicle route is derived for the Euclidean, Manhattan and
% hyperbolic metrics. The main result is that the effect of walking is strongly dependent on the sharpness of turns in
% a vehicle route: if all customers are located along the same road, no advantage is gained by redirecting. 
% On the other hand, if the service is near door to door, the distance driven by
% the vehicle may be reduced up to 2 times the total walking distance of customers.
% 
% In chapter \ref{simutools}, an immediate request dynamic vehicle routing problem with pick-ups and deliveries is studied.
% The focus is on systems where a large number of vehicles are needed to support
% the transportation demand. As a particular feature of the system, a vehicle is assigned to
% each passenger immediately upon the trip request. Simulation experiments are used to show that 
% that in this context it is typically sufficient, without any significant loss in performance, to consider
% the insertion approach for route enumeration, where the relative order
% of the earlier waypoints is always kept the same. 
% That is, it is not necessary to enumerate
% {\em all} feasible orders of waypoints per trip request and per
% vehicle, which indeed can take some time in a large system. % with high load.
% On the other hand, the viability of this type
% of transportation system is demonstrated by means of simulation experiments. In general, there is a well-known
% trade-off between the work conducted (driven kilometers)
% and the level of the service (e.g., mean waiting times).
% However, the experiments suggest that if the passengers are
% willing to accept even a small average delay for their trips,
% in form of waiting time and/or a longer route, then
% the amount of work can be reduced considerably. That is, the
% transportation cost per trip can be reduced significantly.

As a final conclusion, it can be stated that there are many theoretical results
that support the technical viability of large scale demand responsive transport.
However, as suggested in \cite{cortes}, one of the key issues in a large scale demand 
responsive service ever becoming a reality, is the
institutional inertia against change in transit paradigms. No models exist that are
directly applicable in finding to what extent a completely new transportation system is possible in real life.
How to model demand for a hypothetical transportation service remains a relatively open question.
In order to be able to access such practical problems, future work calls for 
real-life pilot services, which would probably give valuable information on the demand for demand-responsive transport.






%% The following commands are for article dissertations, remove them if you write a monograph dissertation.

% Errata list, if you have errors in the publications.
%\errata

\bibliographystyle{plain}
\bibliography{vk3}


%% The first publication (journal article)
% Set the publication information.
% This command musts to be the first!
\addpublication{Lauri H\"ame}{An adaptive insertion algorithm for the single-vehicle dial-a-ride problem with narrow time windows}{European Journal of Operational Research}{209, p. 11–22}{February}{2011}{Elsevier B.V.}{jeadarp}
% Add the dissertation author's contribution to that publication (the order can be interchanged with \adderrata).
\addcontribution{This article was written by the author.}
% Add the errata of the publication, remove if there are none (the order can be interchanged with \addauthorscontribution).
%\adderrata{This is wrong}
% Add the publication pdf file, the filename is the parameter (must be the last).
\addpublicationpdf{articles/eadarpejor.pdf}

%% The second publication (conference article, note the optional parameter)
% Set the publication information.
\addpublication[conference]{Esa Hyyti\"a, Lauri H\"ame, Aleksi Penttinen, Reijo Sulonen}{Simulation of a Large Scale Dynamic Pickup and Delivery Problem}{SIMUTools}{Malaga, Spain}{March}{2010}{ICST}{csimutools}
% Add the dissertation author's contribution to that publication.
\addcontribution{Parts of this paper were written the author, including most of Section 3 and parts of Section 1. 
The simulations reported in Section 3 were designed and conducted by the author.}
% No errata
% Add the publication pdf file, the filename is the parameter.
\addpublicationpdf{articles/simutools-2010b.pdf}

\addpublication[conference]{Lauri H\"ame, Jani-Pekka Jokinen, Reijo Sulonen}{Modeling a competitive demand-responsive transport market}{Kuhmo Nectar Conference on Transport Economics}{Stockholm, Sweden}{June-July}{2011}{No copyright holder at this moment}{ccompejor}
\addcontribution{The author was the main author of this article, which was nominated for the best student paper award in Kuhmo Nectar 2011.
The article is under review for publication in Economics of Transportation.}
\addpublicationpdf{articles/compejor.pdf}

\addpublication[conference]{Jani-Pekka Jokinen, Lauri H\"ame, Esa Hyyti\"a, Reijo Sulonen}{Simulation Model for a Demand Responsive Transportation Monopoly}{Kuhmo Nectar Conference on Transport Economics}{Stockholm, Sweden}{June-July}{2011}{No copyright holder at this moment}{cmonop_ecotran}
\addcontribution{Parts of this paper were written the author, including major parts of Sections 2 and 3.
The market mechanisms were programmed into the simulation model and the simulations were executed by the author.
The author produced the figures in this paper and the idea of using the simulation model reported in 
"Simulation of a Large Scale Dynamic Pickup and Delivery Problem" to study market mechanims was originally suggested by the author.
This paper is also under review for publication in Economics of Transportation.
}
\addpublicationpdf{articles/monop_ecotran.pdf}

%% The third publication (another journal paper, accepted for publication, note the optional parameter)
% Set the publication information, detailed information can be empty
\addpublication[conference]{Teemu Sihvola, Lauri H\"ame, Reijo Sulonen}{Passenger-Pooling and Trip-Combining Potential of High-Density Demand Responsive
Transport}{Annual Meeting of the Transportation Research Board}{Washington, D.C.}{January}{2010}{Transportation Research Board}{cpooling}
% Add the dissertation author's contribution to that publication.
\addcontribution{The author programmed and executed the simulations reported in this paper.}
% Add the errata of the publication, remove if there are none.
%\adderrata{This is wrong}
% Add the publication pdf file, the filename is the parameter.
\addpublicationpdf{articles/pooling.pdf}


%% The fourth publication (yet another journal paper, submitted for publication, note the optional parameter)
%% Note that you are allowed to use this option only when submitting the dissertation for pre-examination!
% Set the publication information, detailed information is not printed
\addpublication[submitted]{Lauri H\"ame, Harri Hakula}{Dynamic journeying under uncertainty}{Under review for publication in European Journal of Operational Research}{}{20.12.2011}{}{No copyright holder at this moment}{jdjuejor}
\addcontribution{The author was the main author of this article.}
\addpublicationpdf{articles/djuejor2.pdf}

\addpublication[submitted]{Lauri H\"ame, Harri Hakula, Saara Hyv\"onen}{Dynamic journeying in scheduled networks}{Under review for publication in IEEE Transactions on Intelligent Transportation Systems}{}{16.1.2012}{}{No copyright holder at this moment}{jtoits}
\addcontribution{The author was the main author of this article.}
\addpublicationpdf{articles/rbrorl2.pdf}

\addpublication[submitted]{Lauri H\"ame, Harri Hakula}{A Maximum Cluster Algorithm for Checking the
Feasibility of Dial-A-Ride Instances}{Under review for publication in Transportation Science}{}{16.1.2012}{}{No copyright holder at this moment}{jhitsdarpts}
\addcontribution{The author was the main author of this article.}
\addpublicationpdf{articles/hitsdarpts2.pdf}

\addpublication[submitted]{Lauri H\"ame, Harri Hakula}{Routing by Ranking: A Link Analysis Method for the Constrained Dial-A-Ride Problem}
{Under review for publication in Operations Research Letters}{}{16.1.2012}{}{No copyright holder at this moment}{jrbrorl}
\addcontribution{The author was the main author of this article.}
\addpublicationpdf{articles/rbrorl2.pdf}





\end{document}
