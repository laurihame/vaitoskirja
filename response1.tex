\documentclass[a4paper,12pt]{article}
\usepackage{longtable,amsmath,amssymb}
\addtolength{\textwidth}{0.5in}
\addtolength{\hoffset}{-0.5in}
\addtolength{\textheight}{0in}
\addtolength{\voffset}{-0.5in}


\begin{document}
\section*{Demand-Responsive Transport: Models and Algorithms}
\subsection*{Statement of corrections}
The responses to the comments are presented in the table below. \\
Lauri H\"ame

\begin{longtable}{|p{0.5\textwidth}|p{0.5\textwidth}|}
\hline
Comment & Response \\ \hline
Chapter 3 solves the problem of journey planning by the passengers using Markov decision
processes (MDP). \ldots I am concerned with two points. First
is the computational complexity. Assume that $m$ is the maximal number of possible next states.
Then the author says that the complexity of performing an iteration is $m$ ln$(m)$ because of the
ordering. However, if each ordering corresponds to an action then there are $m!$ possible actions,
and this must affect the complexity, but I do not see this in the complexity evaluation in the
thesis. It has to be made clear how the candidate got around this problem. 
& 
In the current version, a note on the complexity has been added (p. 41). Although the total number of 
possible actions for a successor set $S_s$ equals $|S_s|!$, the complexity of the algorithm is smaller due to the fact that 
the successors of each state are ordered only once. That is, the algorithm goes through all states reachable from the origin state 
but not through all possible actions. 
\\ \hline
Second, starting from
Section 3.2.4, the ranking by expected number of paths has been introduced, giving a larger value
to the states that have more possible routes to the destination. Then in numerical experiments
the author computes the probabilities of reaching the destination. It is not clear how this affects
the travel times. Intuitively, from the model, it seems that the passengers are encouraged to hop
from one hub to another, and the travel times are ignored. This does not look realistic, and must
be better explained to the reader. 
& 
In the current version (p. 43) it is noted that ``The numerical examples are restricted to calculating the the expected number of feasible paths and 
the probability of reaching the destination during a pre-defined time horizon. However, a general objective function
defined as a combination of expected waiting time, number of transfers, expected ride time, expected walking time and reliability
is incorporated with minimal effort (see Equation (6) in Publication III).'' Travel times are not ignored even in the numerical 
experiments, since the probability of reaching the destination during the time horizon decreases when travel time increases.
\\ \hline
In Theorem 3, ‘of L’ must be ‘of P ’ (same typo appears in
Publication III).
& 
Theorem 3 has been corrected and an errata has been added.
\\ \hline
First, the chapter is difficult to understand without reading Publication IV, because many terms 
are not clear or not defined (for instance ‘combination of price and
waiting time’), but even after reading the publication many terms remain unclear (for example,
‘customer surplus’). Also, the important features of the model should be directly emphasized (for
example, that the customers may choose not to use the service so $Q_{ij}^{DRT}$ is not necessarily 
equal to $Q_ij$ ). 
& 
I have made an effort to clarify and define the terms used in Chapter IV. The combination of ticket price and travel time
is explained in Section 4.1 (p. 46) and a definition of customer surplus has been added to Section 4.3.3 (p. 49).
The important features have been emphasized, for example, 
``$Q_{ij}$ denotes the total demand from $i$ to $j$, including the demand $Q_{ij}^{DRT}$ for DRT and the demand for the virtual mode.'' (p. 46).
\\ \hline
Second, the state is defined as the remaining route, and thus there are many states even
in the simple three-node case. In a larger network, such definition of the state will result in the
state space explosion. It will be helpful to include at least some comments in this regard. 
& 
The definition of states has been commented in Section 4.3.2 (p. 48):
``In large networks, one might want to
limit the number of states by including only a part of all 
possible sequences, since the number of possible sequences of $n$ nodes equals $n!$. ''
\\ \hline
Finally,
the author says repeatedly that the model is similar to the one in [85]. Then I suggest to state
clearly what are is novel in Chapter 4 compared to [85].
& 
The novelty in Chapter 4 has been stated in the beginning of the Chapter (p. 45):
``The main difference is that in a taxi service [85], 
customers are delivered to their destinations directly, whereas in a demand-responsive transport service, a vehicle can serve several customers 
simultaneously and therefore a customer's trip from an origin to a destination is not necessarily a direct one. 
In other words, the route of a taxi is determined by two points (origin and destination), but 
demand-responsive transport routes may include several stops, similarly as bus routes.''

\\ \hline
The Conclusions is the weakest part of the thesis. The statement that there is a theoretical
evidence for advantages of the DRT does not sound convincing in the current form, I suggest to
support it with an example, preferably from the thesis itself. 
& 
This part has been modified to the following form: ``\ldots many computational results
that support the technical viability of demand-responsive transport. State-of-the-art 
algortihms are capable of efficiently solving complex routing problems with multiple vehicles (see for example Table 2.2. in Chapter 2, p. 33).''
\\ \hline
Next, I think the Conclusions will
be stronger if the candidate lists innovative aspects of the thesis and the main contributions into
scientific methodology. 
& 
A list of innovative aspects and main contributions has been added to the conclusions (p. 53-54).
\\ \hline
Also, I suggest to add a well-grounded discussion of future research.
& 
A discussion of future research has been added to the conclusions (p. 54).
\\ \hline
There is a small number of typo’s in the thesis, one more round of proofreading will help to
eliminate those.
& 
Corrected.
\\ \hline
\end{longtable}
\end{document}

