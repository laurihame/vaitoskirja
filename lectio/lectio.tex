\documentclass[a4paper,12pt]{article}
\usepackage[finnish]{babel}
\usepackage[utf8]{inputenc}
\addtolength{\textwidth}{0.5in}
\addtolength{\hoffset}{-0.5in}
\addtolength{\textheight}{0in}
\addtolength{\voffset}{-0.5in}


\begin{document}
\section*{Lectio Praecursoria 31.5.2013}
Arvoisa valvoja, arvoisa vastaväittäjä, arvoisat kuulijat.

%Esittelen väitöskirjani, jonka otsikko on ``Kysyntäohjautuvan joukkoliikenteen matemaattisia malleja ja algoritmeja''. 
Kysyntäohjautuvalla joukkoliikenteellä tarkoitetaan bussi- ja taksipalvelujen välimuotoa, joka perustuu pienten tai 
keskisuurten ajoneuvojen joustavaan reititykseen. Kysyntäohjautuvassa joukkoliikenteessä matkat tilataan etukäteen ja
ajoneuvojen reitit muodostuvat matkatilausten perusteella.
Väitöskirjassa on tutkittu ja kehitetty matemaattisia malleja kysyntäohjautuvalle joukkoliikenteelle, ja algoritmeja eli menetelmiä,
joilla voidaan ratkaista ajoneuvojen reitinlaskentaan ja matkansuunnitteluun liittyviä kombinatorisia ongelmia liikenneverkossa.

Väitöskirjan ensimmäinen osa käsittelee ajoneuvojen reitinlaskentaongelmaa, kun oletetaan kysyntä tunnetuksi yhden matkustajan tarkkuudella. 
Toisessa osassa tarkastellaan matkustajien matkansuunnittelua, joka liittyy joukkoliikennevälineen ja reitin valintaan joukkoliikenneverkossa.
Lopuksi tarkastellaan taloudellisen tasapainopisteen, eli kysynnän ja tarjonnan kohtaamispisteen, määrittämistä liikenneverkossa.

Reitinlaskentaongelmista tunnetuin on niin sanottu kauppamatkustajan ongelma. Ongelman määrittelee joukko maantieteellisiä 
pisteitä, esimerkiksi kaupunkeja, joiden väliset etäisyydet tunnetaan. Tavoitteena on löytää lyhin reitti joka kulkee kaikkien pisteiden kautta.
Kauppamatkustajan ongelma on laskennallisesti haastava: sen ratkaisemiseen tunnetaan ainoastaan algoritmeja, joiden laskenta-aika kasvaa 
eksponentiaalisesti pisteiden määrän suhteen.

Käytännössä, esimerkiksi kuljetuspalveluissa, reitinlaskentaongelma on usein kauppamatkustajan ongelmaa monimutkaisempi. 
Reitinlaskentaan voi liittyä erityyppisiä rajoituksia, ja lisäksi jos ajoneuvoja on useita, nouto- ja toimituspisteet voidaan 
jakaa usealle reitille.

Kapasiteettirajoituksilla tarkoitetaan sitä, että 
ajoneuvoihin mahtuu vain tietty määrä kuljetettavaa tavaraa tai matkustajia kerrallaan. Aikarajoituksilla huolehditaan siitä että
tavaran tai matkustajan kuljetus ei kestä liian kauan. Edeltävyysrajoitukset tarkoittavat kuljetuksessa sitä että tavaran
tai matkustajan noutopisteessä pitää käydä ennen toimituspistettä. 

Kysyntäohjautuvassa joukkoliikenteessä asiakkaat voivat tilata matkoja reaaliaikaisesti esim. internet-käyttöliittymällä ja 
ajoneuvojen reitit muodostuvat tilattujen matkojen perusteella. Jokaiselle uudelle asiakkaalle valitaan ajoneuvo ja 
valitulle ajoneuvolle määrätään uusi reitti. Ajoneuvon ja reitin valinnassa tulee ottaa huomioon muun muassa 
uuden asiakkaan aiheuttama reitin pitenemä, uuden asiakkaan palvelutaso ja muille asiakkaille aiheutuva palvelutason muutos.
%Jos ainoastaan minimoidaan reitin pituutta, palvelutaso saattaa kärsiä ja jos optimoidaan ainoastaan palvelutasoa, palvelun 
%tuotantokustannukset kasvavat.

Ajoneuvon ja reitinvalintaongelma voidaan ratkaista joko hajautetusti tai keskitetysti. Hajautetussa ratkaisussa jokaiselle ajoneuvolle
lasketaan uusi reittiehdotus






Väitöskirjan tärkeimmät tulokset tieteellisen metodologian näkökulmasta liittyvät reitinlaskentaan muuttuvissa
Erityisesti matkustajaliikenteeseen




Pyydän teitä, arvoisa professori, Aalto-yliopiston perustieteiden korkeakoulun määräämänä vastaväittäjänä esittämään ne muistutukset, joihin katsotte väitöskirjan antavan aihetta.


\end{document}
